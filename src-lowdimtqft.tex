We now start to work out the structure of low dimensional ordinary TQFTs to get an idea of how a classification might look in general. To this end we return to oriented cobordisms for the next sections. We start with $1$-dimensional TQFTs in section \ref{sec:1dimtqft} which are particularly easy to classify. In section \ref{sec:2dimtqft} we then consider $2$-dimensional TQFTs for which an explicit classification in terms of algebraic objects is still not too difficult but already involves much more data.
