%\nocite{00000001}
%\nocite{wiki-nlab0000}
%%%
Quantum field theory (QFT) is the framework in which fundamental processes in physics, especially in particle physics, are described today. However, it is not that easy to say what a QFT\footnote{note that the term quantum field theory can mean a single quantum field theory or the whole subject of quantum field theories (QFTs), but it should always be clear from the context what is meant} is as there is no canonical, generally agreed upon definition. Historically quantum field theory began with the description of the electromagnetic field in a more quantum kind of way by treating it as a collection of quantum harmonic oscillators. This was necessary because in ordinary, or relativistic\footnote{The Schr{\"o}dinger equation of ordinary QM is not compatible with special relativity since it treats time and space in a fundamentally different fashion. There is, however, a relativistic version of QM whose guiding equation is the Dirac equation which can be viewed as a relativistic improvement of the Schr{\"o}dinger equation.}, quantum mechanics (QM) the interaction of particles like electrons with the electromagnetic field is described semi-classically, i.e. the particles are treated in a quantum mechanical way but the electromagnetic field is treated classically, and this description is unable to capture the spontaneous emission of a photon by an excited\footnote{excited means that the particle is in a state with higher energy than the ground state which is the state of minimal energy} electron. The quantum mechanical description of the electromagnetic field indeed is able to account for spontaneous emission and led to the precursor of quantum electrodynamics (QED). This precursor is not really a QFT since at that time the material particles were viewed as eternal, which means that they cannot be created or destroyed, whereas photons were considered as excited states of the electromagnetic quantum field offering the possibility of their creation and destruction. But soon it was realized that material particles can be described by quantum fields in a similar way, so that, given enough energy, material particles can come into and out of existence, too, which is in contrast to QM where one deals with a fixed (usually small) number of particles. The description of electrons as a field finally led to QED, the first real QFT.
\\\\
So, QFT emerged from the attempt to combine classical field theory (CFT) with quantum mechanics in the context of special relativity. More generally, one also wants to consider curved spacetimes\footnote{we can think of spacetime as a Lorentzian manifold with a time-orientation, which roughly is a semi-Riemannian manifold with one dimension for time and the other dimensions for space in a proper sense} and not only Minkowski space, which is necessary for example to describe radiation of black holes. For a sketch of the basic idea which we aim to describe in the following this will not play much of a role. Note however that in the general case of curved spacetimes the concept of particles is generally not sensible anymore. Moreover, we should briefly talk about fields. Naively a field or field configuration or field history\footnote{a field configuration is sometimes rather understood as being defined only on space and not on spacetime, so field history is probably the better terminology} is a smooth function from spacetime to some other space, usually a manifold. Scalar fields, for example, have values in the field one considers, usually the real or complex numbers. In general, however, fields are not defined globally but rather in a local fashion which basically means that they are actual functions only on the local charts of spacetime. The principle of locality then says that the global fields should be determined by the local data. Yet, according to the gauge principle, one should not simply identify the local data in the overlapping region of two coordinate charts but there should be a gauge transformation connecting them. In this way one can capture topologically non-trivial global information. To properly describe this, one uses certain fiber bundles so that fields are really sections of fiber bundles. This will not play much of a role for the following (very basic) outline of QFT, at least not explicitly, and the reader is referred to \cite{00000001} for more on this.
\\\\
The basic idea in QFT is to start with a classical Lagrangian (density\footnote{the Lagrangian itself is actually often understood to be the volume integral of the Lagrangian density over space but for brevity we use the word Lagrangian for Lagrangian density}) $\mathcal{L}$ describing all the fields in a classical fashion. Remember that the Lagrangian is a real-valued function of the fields and their derivatives (usually only finitely many, e.g. only the first derivatives) and integrating $\mathcal{L}$ over spacetime yields the action (functional) of the system. There may be terms for a variety of fields in such a Lagrangian, that is, the Lagrangian generally is a finite sum
\begin{align*}
  \mathcal{L}
  &=
  \mathcal{L}_{1}
  +
  \mathcal{L}_{2}
  +
  \ldots
  +
  \mathcal{L}_{n}
\end{align*}
of different contributions. The term for the field of the electron\footnote{or other spin-$\frac{1}{2}$ fermions}, for example, comes from the Dirac field obtained by {\glqq}reinterpreting{\grqq} the general solution to the corresponding wave equation from relativistic QM, the Dirac equation, as a classical field. This means that the Euler-Lagrange equation for this term is exactly the Dirac equation. A scalar field is described by a term corresponding to the Klein-Gordon equation, another equation in relativistic QM. Vector fields, like the electromagnetic field, may be described using field strength tensors. In addition to these terms for free fields one needs terms for the interaction of the free fields to obtain realistic QFTs. These are usually appropriate products of the fields one intends to couple, multiplied with some coupling factor controlling the strength of the interaction. The Lagrangian thus is often written as
\begin{align*}
  \mathcal{L}
  &=
  \mathcal{L}_{\textrm{free}}
  +
  \mathcal{L}_{\textrm{int}}
\end{align*}
where $\mathcal{L}_{\textrm{free}}$ is the part for the free fields and $\mathcal{L}_{\textrm{int}}$ is the interaction part. An important way to obtain the correct interaction terms is by contemplating gauge symmetries of the system considered, which in terms of Lagrangians means that the Lagrangian is demanded to be invariant under certain gauge transformations. Such theories are called gauge theories. Starting, for example, with a free Dirac field for the electron one can obtain all other terms in the Lagrangian for QED by applying gauge theory with gauge group\footnote{this is basically the group of gauge transformations} $U(1)$. One important systematic way to obtain the correct terms in this way (especially for non-abelian gauge groups) is Yang-Mills theory. For a more accurate account on gauge theories we refer the reader to \cite{00000001}. Having all the terms in the Lagrangian one can obtain the Hamiltonian and the generalized momentum fields corresponding to the fields as usual by Legendre transformation.
\\
Now, the next step is to quantize the fields. To understand what quantization is supposed to mean let us first look at quantum mechanics. To arrive at quantum mechanics from classical mechanics in its Hamiltonian formulation, one intends to replace the classical observables\footnote{observables are meant to be the quantities that are measurable, at least in principle, like the position and momentum of a particle or the energy of the system} by operators on a Hilbert space of states of the quantum system. Observables in Hamiltionian mechanics are functions from the phase space to the real or complex numbers, so they form an algebra by using the pointwise operations, say $(A,\cdot)$ where $A$ is the underlying vector space and $\cdot$ the pointwise product. The phase space is the space of states in Hamiltonian mechanics and usually consists of all possible values of the generalized position and momentum variables of the system. Hence the obervables are functions of the generalized position and momentum variables and in particular these variables are turned into operators when doing the quantization. The obtained operators for position and momentum are then required to satisfy certain so-called canonical commutation relations - which encode the famous uncertainty principle - instead of the corresponding relations for the Poisson bracket of classical observables. Actually, one needs to impose other conditions in order to make this process of quantization sensible but the idea should be clear.
\\
The idea of quantization in QFT is basically the same. The fields and their corresponding momentum fields now become operators (at any spacetime point) acting on quantum states in a Hilbert space and they are again required to satisfy a field version of the canonical (anti\footnote{in the case of fermions one has anti-commutators instead of commutators}-)commutation relations. The Hamiltionian after quantization is then used to describe the time evolution of the system. Because of the canonical relations satisfied by the fields and momentum fields this quantization process is usually called canonical quantization. As some of the fields, like the Dirac field, have been obtained from quantum mechanical wave equations people sometimes also speak of second quantization (for historical reasons) when quantizing fields. Yet this is not really a good terminology because these fields were obtained by a reinterpretation of the wave functions of quantum wave equations as classical fields so that they are not quantized. For free field theories, i.e. without any interaction terms, one can indeed obtain QFTs using this idea. However, for interacting theories this is very difficult and only very simple examples have been worked out.
\\
There is an alternative approach to quantization using the so called path integral. For this approach the classical Lagrangian is used to calculate the action for the system. In classical Lagrangian mechanics the evolution of the system is obtained by examining where this action is stationary, i.e. where its functional derivative is zero. The idea of the path integral is then that not only the field which makes the action stationary determines how things evolve but one has to integrate over all fields, where each field is weighted with a phase factor determined by the action. This approach is often more convenient to work with, which is why it is often used by physicists today. Another advantage is that this approach is compatible with special relativity in an obvious way as opposed to the approach of canonical quantization for which this is not so obvious. However, there is a major problem: even though this approach yields physically sensible results, for most QFTs the path integral does not exist in a mathematically precise sense since there is no measure allowing to integrate over all fields.
\\\\
From a mathematical point of view there are essentially two approaches towards quantization. Before describing them we note that the phase space of a classical system is given by a sympletic manifold $(E,\omega)$, that is, a manifold $E$ together with a closed non-degenerate $2$-form $\omega$, called a sympletic form. This $2$-form determines a Lie bracket $\lbrace -,- \rbrace$, called the Poisson bracket, which controls the dynamics of the system and makes the commutative algebra of classical observables $(A,\cdot)$ into a Poisson algebra $(A,\cdot,\lbrace -,- \rbrace)$. This means in particular\footnote{there is another condition a Poisson algebra satisfies by definition which in some sense encodes the compatibility of the commutative product and the Poisson bracket} that the observables as a vector space together with the Poisson bracket form a Lie algebra $(A,\lbrace - ,- \rbrace)$. Now the two approaches for quantization are
\begin{itemize}
\item
Deformation quantization: this approach to quantization focuses on the algebra of observables and hence on the Heisenberg picture in which the dynamics of the system is encoded in the observables. The basic idea is to deform the commutative algebra of classical observables $(A,\cdot)$ by taking the same vector space but endowing it with a new product, denote it $\star$, which makes this vector space into a non-commutative algebra of quantum observables $(A,\star)$. The new product is required to have the property that the resulting commutator
\begin{align*}
  [O_{1},O_{2}]_{\star}
  &=
  O_{1}
  \star
  O_{2}
  -
  O_{2}
  \star
  O_{1}
\end{align*}
for two observables $O_{1},O_{2} \in A$ is to the {\glqq}first order{\grqq}, in an appropriate sense, determined by the Poisson bracket $\lbrace -,- \rbrace$. So, in some sense one deforms the commutative product to a non-commutative product in such a way that the commutator is a deformation of the Poisson bracket. Hence the name.
\\
Traditionally, and also often today, deformation quantization means formal deformation quantization which yields an algebra of formal power series $A[[\hbar]]$ in a formal parameter, usually denoted $\hbar$ which physically of course means Planck's constant. Thus the new product $\star$ is not really defined on the vector space $A$ of the original algebra but on the larger algebra of formal power series $A[[\hbar]]$ with coefficients in the original algebra. In the limit $\hbar \to 0$ this deformation should reproduce the original algebra. Formal deformation quantization actually corresponds to perturbative QFT (see below) so that the formal power series are usually cut off at some finite order in practice.
\\
A {\glqq}genuine{\grqq} quantization is not expected to yield just formal power series but actual functions. This is called strict deformation quantization or $C^{\ast}$-algebra deformation quantization because the resulting algebra is supposed to be a so-called $C^{\ast}$-algebra. 

\item
Geometric quantization: this approach focuses on the space of states and hence on the Schr{\"o}dinger picture in which the dynamics of the system is encoded in the states. Geometric quantization proceeds in two steps:
\begin{enumerate}
\item
The first step goes by the name prequantization. One chooses a certain line bundle\footnote{that is, a vector bundle of rank $1$} $L$ over $E$ with a connection\footnote{a connection basically says how the fibers of the vector bundle are connected and gives a way to identify fibers via a parallel transport which allows to do differentiation; for more on connections see e.g. \cite{00000001}}\,\footnote{more precisely one takes a line bundle with a bundle metric - hence one has an inner product for every fiber - and one takes a $U(1)$ connection - hence the corresponding parallel transport preserves the inner product} whose curvature $2$-form is basically the symplectic form $\omega$ of the given sympletic manifold $E$ representing the phase space of the classical system. This is called the prequantum line bundle. Under a certain condition for $\omega$ this is uniquely determined up to isomorphism. The (square-integrable) sections of this line bundle then form a Hilbert space\footnote{one can define an inner product on the space of sections of the line bundle by using the symplectic structure of the base manifold to define a volume form and hence do an integration; a section whose inner product with itself is finite is then called square-integrable} $H_{\textrm{p}}$ called the prequantum Hilbert space. One can also prequantize the classical observables by a map sending an observable to an operator on the prequantum Hilbert space in such a way that these operators satisfy commutator relations corresponding to the Poisson bracket relations of the corresponding classical observables. This involves the covariant derivative coming from the connection of the prequantum line bundle.

\item
However, the prequantum Hilbert space $H_{\textrm{p}}$ is too big to be the actual Hilbert space of quantum states. To cut it down one chooses a so called polarization which roughly is a way to divide the symplectic manifold $E$ representing the classical phase space into {\glqq}coordinates{\grqq} and {\glqq}momenta{\grqq}. The quantum Hilbert space $H \subset H_{\textrm{p}}$ then basically is the subspace of the prequantum Hilbert space consisting of those (square-integrable) sections of the prequantum line bundle $L$ which only depend on the coordinates and not on the momenta. Finally, one aims to take the classical observables to operators on this quantum Hilbert space $H$ in such a way that they satisfy appropriate commutator relations as can be done at the prequantum level. Unfortunately this is not possible for all observables on the quantum level, so things are a bit trickier here and one has to restrict to a certain subset of observables.
\end{enumerate}
\end{itemize}
In full generality the process of quantization is not very well-understood thus far. In fact, the (non-perturbative) quantization of Yang-Mills theory, which plays a fundamental role in the standard model of particle physics (SMP), is one of the famous {\glqq}Millenium Problems{\grqq}. But things become significantly easier when considering perturbative quantum field theory (pQFT). As much of the activity in physics is actually concerned with pQFT it is in order to briefly outline what the idea is.
\\\\
Perturbative QFT is meant to be an easier-to-handle approximation to full non-perturbative QFT. In pQFT one treats a part of the theory, usually the interaction part, as a small perturbation of the rest, usually the free fields. The time evolution in QFT is determined by an exponential of the Hamiltonian - or the Lagrangian in the path integral formulation - and the idea is to expand the interaction part\footnote{or more generally the part that represents the perturbation} of this exponential as a Taylor series in the coupling factor of the interaction which is assumed to be small. In this way the observables become (formal) power series\footnote{they are usually meant to be asymptotic expansions, i.e. they do not converge but still can be truncated at any finite order to yield controlled approximations to a function} in the coupling factor (and actually also in Planck's constant) and since this factor is supposed to be small it suffices to only consider low order terms to obtain good results.
\\
One is chiefly concerned with calculating so-called correlators in pQFT\footnote{and actually also in full QFT} which are basically (vacuum) expectation values\footnote{the vacuum expectation value can in principle be thought of as the average value for a measurement in the state with the lowest possible energy (vacuum)} of appropriate products of fields at some, potentially different, spacetime points. More precisely, one often deals with scattering amplitudes which are the probability amplitudes for scattering processes of particles or more generally of fields\footnote{in physics one mostly considers pQFT on Minkowski space so that the concept of particles as excitations of the fields makes sense}. One can think of this as a process where freely propagating fields in some state come in from the far past, then interact, i.e. scatter off each other, and then go off in some other state into the far future\footnote{in real applicatons the far past and far future are not that far on a classical level but compared to the {\glqq}time scale{\grqq} of the interaction this is a good approximation}. The scattering amplitudes are organized in the so called S-matrix which is then used to calculate numerical predictions - such as scattering cross sections - and is pretty much the key object of perturbative QFT. The scattering cross sections can be measured in scattering experiments, as done e.g. at the LHC particle accelerator, which are the main source of experimental verification of quantum field theory, or more particularly the SMP, today. Indeed, the predictions made with the help of perturbative methods about the SMP are in agreement with experiment to a (sometimes very) good precision. Due to the perturbative nature of the theory the S-matrix can be expressed as a sum over elementary processes, that is, specific ways of interaction. This leads to the famous Feynman diagrams governed by the Feynman rules.
\\
However, these computations, as physicists often do them, may lead to infinite values ({\glqq}divergences{\grqq}) of some quantities which is why one has to apply a process called renormalization. The problem basically is that there are some interactions that are not accounted for by the perturbative approach and thus some corrections are necessary. Moreover, physicists often resort to the path integral when doing calculations, which, as said above, cannot be defined rigorously in most cases. But there also is a mathematically rigorous formulation of pQFT properly treating these problems. The main idea is to focus on the S-matrix and to axiomatize how it should behave. The key property is called causal additivity and basically means that all effects caused in some spacetime region must be in the causal future of that region. Therefore this rigorous formulation of pQFT goes by the name causal perturbation theory. There also is a slight refinement called perturbative AQFT (pAQFT) where the observables are organized in such a way that they satisfy a variation of the Haag-Kastler axioms, which are an attempt to axiomatize QFT in general (see below), hence the name. This is in fact a rigorous formulation of perturbative QFT on Minkowski space, justifying often-used tools such as the Feynman diagrams. There are other equivalent rigorous formulations for pQFT on Minkowski space but the above approach by pAQFT has the advantage that it easily generalizes to more general curved spacetimes, at least to globally hyperbolic spacetimes which are an important subclass. This generalization is now called locally covariant perturbative AQFT (lcpAQFT). While mathematically precise and well motivated by heuristics of the path integral and also by experimental confirmation it seems desirable, at least from a physical point of view, that one can derive these constructions by some kind of quantization. As it has turned out quite recently this is indeed the case for (at least parts of) lcpAQFT, specifically by using some kind of formal deformation quantization called Fedosov deformation quantization. A (very) comprehensive introduction to lcpAQFT is given in the article \href{https://ncatlab.org/nlab/show/geometry+of+physics+--+perturbative+quantum+field+theory}{geometry of physics - perturbative quantum field theory} in \cite{wiki-nlab0000}. Now, apart from only being perturbative, there is basically only one problem left with lcpAQFT, namely when it comes to gauge theory on curved spacetimes one of the axioms encoding locality breaks. But there already is a proposed solution called homotopical AQFT where one passes to so called homotopical algebras of observables. But this is still in the making.
\\\\
While pQFT is comparatively well understood, though already quite involved mathematically, there are also non-perturbative effects (known e.g. from computer simulations or experiment) which cannot be captured with perturbative methods. A notable example is the confinement of quarks in quantum chromodynamics (QCD) at low energy. For low energies the coupling factor in QCD becomes large so that perturbation theory cannot be expected to be a good approximation anymore. An important keyword for non-perturbative effects is \textit{instantons}. These encode topologically non-trivial information of fields which may be a source for non-perturbative effects. There are methods like resurgence theory trying to capture non-perturbative effects in pQFT but it is far from clear whether this works. Thus a rigorous understanding of pQFT is only a step towards understanding full non-perturbative QFT.
\\
Non-perturbative QFT has remained much more elusive than its perturbative approximation, especially when it comes to quantization. As already mentioned above quantization of full QFT is currently not very well understood. Indeed, the only examples constructed satisfying the axioms of an axiomatization of QFT so far are toy models in that they do not incorporate interactions or are on spacetimes of dimension less than $4$. But what does an axiomatization of QFT look like? There are two main approaches for the axiomatization and formalization of QFT and they go by the names algebraic quantum field theory (AQFT) and functorial quantum field theory (FQFT).
\begin{itemize}
\item
AQFT focuses on the observables and thus can be viewed as an axiomatization of the Heisenberg picture. This approach is more widely considered among mathematical physicists. The main idea is to assign (usually $C^{\ast}$-)algebras of observables to certain patches of spacetime in a consistent way, meaning more precisely that they form a so called local net of  observables which is characterized by the Haag-Kastler axioms. In literature the axioms vary a bit but the assignment is always encoded in a presheaf and one of the main axioms is causal locality, basically meaning that observables defined on spacelike-seperated regions of spacetime commute with each other. We do not want to go into further detail here.

\item
FQFT focuses on the quantum states and thus can be viewed as an axiomatization of the Schr{\"o}dinger picture. This is the more recent approach. The main idea is to assign
\begin{itemize}
\item
a vector space $Z(S)$ (possibly a Hilbert space) of quantum states to every codimension-$1$ slice $S$ of spacetime representing space

\item
a linear operator $Z(M) \colon Z(S_{1}) \to Z(S_{2})$ between state spaces - thought of as the time evolution operator - to any (part of) spacetime $M$ which is a cobordism between the spatial slices $S_{1},S_{2}$ corresponding to the state spaces $Z(S_{1}),Z(S_{2})$; such a cobordism $M$ can be thought of as a manifold whose boundary consists of two parts, one for each of the spatial slices $S_{1}$ and $S_{2}$
\end{itemize}
This assignment is required to satisfy certain properties physicists expect the path integral to have, so that one can think of FQFT in some sense as axiomatizing the path integral yet without saying what the integration really is. These properties can be encoded by demanding the assignment to be a functor $Z$ from the category of cobordisms to the category of vector spaces. We will motivate this a little further in a later chapter.
\\
However, there is a problem with these axioms, namely that they break so called general covariance because spacetime is split into space and time which is not to be expected from a {\glqq}reasonable{\grqq} physical theory. What one would rather expect is that it is possible to {\glqq}propagate{\grqq} in spatial directions just as in time direction. Thus one has to consider not only submanifolds of spacetime of codimension $1$ but also of higher codimension. This is what is done in extended FQFT where one considers an extended category of cobordisms which contains manifolds of higher codimensions and hence is naturally cast in the language of higher categories. An extended FQFT is a higher functor from this higher category of cobordisms to an appropriate higher category of vector spaces, though it does not really seem to be clear yet what the {\glqq}correct{\grqq} higher category of vector spaces is. As this is a more local description - in fact corresponding to the locality a QFT is expected to have - it sometimes also goes by the name local FQFT.
\end{itemize}
In principle FQFT and AQFT should describe the same thing from different viewpoints, so they should be dual to each other just as algebra and geometry are dual to each other\footnote{the keyword here is Isbell duality}. In special cases this has already been made precise but both FQFT and AQFT are still in progress and so is their precise relation.
\\\\
A special case of FQFT for which there has been quite some progress and from which the idea of FQFT originally arose is topological QFT (TQFT). We will focus on this special case in this work. Basically, a QFT is called topological if the essential things like observables do not depend on the metric of the spacetime. Thus the manifolds in the cobordism category are not equipped with a metric (or other non-topological structure). On Minkowski space TQFTs are not very interesting because Minkowski space is contractible, that is, topologically trivial. Therefore one is usually concerned with more general curved spacetimes here. A notable example of TQFT is Chern-Simons theory which has become kind of a poster-child of (extended) TQFT. Potential applications of TQFT are topological insulators, the quantum Hall effect and topological quantum computing, but there remains much to be understood. Moreover, there is a classification result for TQFTs, which also is of considerable mathematical importance, namely the cobordism hypothesis. This is rather involved and a semi-formal formulation of the cobordism hypothesis will be at the end of our text.
\\\\
A good first mathematical reference for many of the notions in the above overview of QFT are the corresponding articles in \cite{wiki-nlab0000} and subsequently the references given there.
