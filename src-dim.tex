\nocite{0a816f4c}
%%%
The first property concerns the dimension of the state spaces $Z(S)$ associated to the smooth oriented closed $(n-1)$-manifolds $S \in \mathrm{ob}_{\mathbf{Cob}_{n}}$ by a TQFT $Z$. As it turns out, these state spaces are finite-dimensional for a TQFT. This can be seen by an easy argument about dual objects using that $\mathbf{Cob}_{n}$ is rigid. We have the following
\\
\begin{cor}
\label{cor:dimstsp}
Let
\begin{align*}
  Z
  \colon
  \mathbf{Cob}_{n}
  &\to
  \mathbf{Vec}_{K}
\end{align*}
be a TQFT. Then the vector space $Z(S)$ is finite-dimensional for every $S \in \mathrm{ob}_{\mathbf{Cob}_{n}}$ and $Z(\overline{S})$ is isomorphic to $Z(S)^{\prime}$, where $Z(S)^{\prime}$ is the dual vector space of $Z(S)$.
\end{cor}
\begin{prf}
Lemma \ref{lem:mfduals} shows that $Z(\overline{S})$ is a dual object of $Z(S)$ since $\overline{S}$ is a dual object of $S$. Thus, $Z(S)$ is finite-dimensional since only these objects in $\mathbf{Vec}_{K}$ have duals according to lemma \ref{lem:dualvec}. Furthermore, since $Z(S)^{\prime}$ is another dual object for $Z(S)$, lemma \ref{lem:dualunique} gives the desired isomorphism between $Z(\overline{S})$ and $Z(S)^{\prime}$.
\\
\phantom{proven}
\hfill
$\square$
\end{prf}
From lemma \ref{lem:mfduals} we know that the evaluation and coevaluation for $Z(\overline{S})$ in $\mathbf{Vec}_{K}$ are given by
\begin{align*}
  \Braket{\cdot,\cdot}
  &:=
  \Phi^{-1}
  \circ
  Z(\mathrm{ev}_{S})
  \circ
  \mathsf{H}(\overline{S},S)
  \colon
  Z(\overline{S})
  \otimes
  Z(S)
  \to
  K
  \\
  \gamma
  &:=
  \mathsf{H}^{-1}(S,\overline{S})
  \circ
  Z(\mathrm{coev}_{S})
  \circ
  \Phi
  \colon
  K
  \to
  Z(S)
  \otimes
  Z(\overline{S})
\end{align*}
from the evaluation $\mathrm{ev}_{S}$ and coevaluation $\mathrm{coev}_{S}$ in $\mathbf{Cob}_{n}$. Fixing a $\bar{v} \in Z(\overline{S})$ we obtain a linear map 
\begin{align*}
  \Braket{\bar{v},\cdot}
  \colon
  Z(S)
  &\to
  K
\end{align*}
from the evaluation by
\begin{align*}
  v
  \mapsto
  \Braket{\cdot,\cdot}(\bar{v} \otimes v)
  &=:
  \Braket{\bar{v},v}
\end{align*}
The linearity of the map is clear from the linearity of $\Braket{\cdot,\cdot}$ and the universal property of the tensor product which yields a unique bilinear map corresponding to $\Braket{\cdot,\cdot}$. Analogously we obtain a linear map
\begin{align*}
  \Braket{\cdot,v}
  \colon
  Z(\overline{S})
  &\to
  K
\end{align*}
for any fixed $v \in Z(S)$. Now let
\begin{align*}
  f
  \in
  \mathrm{mor}_{\mathbf{Vec}_{K}}
  \left(
    Z(\overline{S})
    ,
    Z(S)^{\prime}
  \right)
\end{align*}
denote the unique isomorphism of dual objects from lemma \ref{lem:dualunique}. For this we have
\begin{align*}
  f(\bar{v})
  &=
  \Braket{\bar{v},\cdot}
\end{align*}
which can be seen as follows: from lemma \ref{lem:dualunique} we know that
\begin{align*}
  f
  &=
  \mathsf{L}(Z(S)^{\prime})
  \circ
  \left(
    \Braket{\cdot,\cdot}
    \otimes
    \mathrm{id}_{Z(S)^{\prime}}
  \right)
  \circ
  \mathsf{A}^{-1}(Z(\overline{S}),Z(S),Z(S)^{\prime})
  \circ
  \left(
    \mathrm{id}_{Z(\overline{S})}
    \otimes
    \mathrm{coev}_{Z(S)}^{\prime}
  \right)
  \circ
  \mathsf{R}^{-1}(Z(\overline{S}))
\end{align*}
Here we denoted the evaluation and coevaluation for $Z(S)^{\prime}$ by $\mathrm{ev}_{Z(S)}^{\prime}$ and $\mathrm{coev}_{Z(S)}^{\prime}$. Let $k$ be the dimension of\footnote{and thus that of $Z(S)^{\prime}$ and $Z(\overline{S})$} $Z(S)$ and let
\begin{align*}
  \lbrace
    b_{1}
    ,
    \ldots
    ,
    b_{k}
  \rbrace
\end{align*}
be a basis  for $Z(S)$ and
\begin{align*}
  \lbrace
    b_{1}^{\prime}
    ,
    \ldots
    ,
    b_{k}^{\prime}
  \rbrace
\end{align*}
the corresponding dual basis for $Z(S)^{\prime}$. Using the linearity of $\Braket{\cdot,\cdot}$ and the bilinearity of the tensor product we calculate
\begin{align*}
  &
  f(\bar{v})
  \\
  &=
  \mathsf{L}(Z(S)^{\prime})
  \circ
  \left(
    \Braket{\cdot,\cdot}
    \otimes
    \mathrm{id}_{Z(S)^{\prime}}
  \right)
  \circ
  \mathsf{A}^{-1}(Z(\overline{S}),Z(S),Z(S)^{\prime})
  \circ
  \left(
    \mathrm{id}_{Z(\overline{S})}
    \otimes
    \mathrm{coev}_{Z(S)}^{\prime}
  \right)
  \circ
  \mathsf{R}^{-1}(Z(\overline{S}))
  \left(
    \bar{v}
  \right)
  \\
  &=
  \mathsf{L}(Z(S)^{\prime})
  \circ
  \left(
    \Braket{\cdot,\cdot}
    \otimes
    \mathrm{id}_{Z(S)^{\prime}}
  \right)
  \circ
  \mathsf{A}^{-1}(Z(\overline{S}),Z(S),Z(S)^{\prime})
  \circ
  \left(
    \mathrm{id}_{Z(\overline{S})}
    \otimes
    \mathrm{coev}_{Z(S)}^{\prime}
  \right)
  \left(
    \bar{v}
    \otimes
    1
  \right)
  \\
  &=
  \mathsf{L}(Z(S)^{\prime})
  \circ
  \left(
    \Braket{\cdot,\cdot}
    \otimes
    \mathrm{id}_{Z(S)^{\prime}}
  \right)
  \circ
  \mathsf{A}^{-1}(Z(\overline{S}),Z(S),Z(S)^{\prime})
  \left(
    \sum_{i=1}^{k}
    \bar{v}
    \otimes
    \left(
      b_{i}
      \otimes
      b_{i}^{\prime}
    \right)
  \right)
  \\
  &=
  \mathsf{L}(Z(S)^{\prime})
  \circ
  \left(
    \Braket{\cdot,\cdot}
    \otimes
    \mathrm{id}_{Z(S)^{\prime}}
  \right)
  \left(
    \sum_{i=1}^{k}
    \left(
      \bar{v}
      \otimes
      b_{i}
    \right)
    \otimes
    b_{i}^{\prime}
  \right)
  \\
  &=
  \mathsf{L}(Z(S)^{\prime})
  \left(
    \sum_{i=1}^{k}
    \Braket{\bar{v},b_{i}}
    \otimes
    b_{i}^{\prime}
  \right)
  \\
  &=
  \mathsf{L}(Z(S)^{\prime})
  \left(
    \sum_{i=1}^{k}
    1
    \otimes
    \Braket{\bar{v},b_{i}}
    b_{i}^{\prime}
  \right)
  \\
  &=
  \sum_{i = 1}^{k}
  \Braket{\bar{v},b_{i}}
  b_{i}^{\prime}
\end{align*}
for $\bar{v} \in Z(\overline{S})$. We further know that
\begin{align*}
  f(\bar{v})
  &=
  \sum_{i = 1}^{k}
  f(\bar{v})(b_{i})
  b_{i}^{\prime}
\end{align*}
and the uniqueness of the basis representation in $Z(S)^{\prime}$ yields
\begin{align*}
  f(\bar{v})(b_{i})
  &=
  \Braket{\bar{v},b_{i}}
  \qquad
  \text{for}
  \qquad
  i
  \in
  \lbrace
    1
    ,
    \ldots
    ,
    k
  \rbrace
\end{align*}
This shows
\begin{align*}
  f(\bar{v})
  &=
  \Braket{\bar{v},\cdot}
\end{align*}
because it suffices to fix linear maps on a basis. We have a basis
\begin{align*}
  \lbrace
    f^{-1}(b_{1}^{\prime})
    ,
    \ldots
    ,
    f^{-1}(b_{k}^{\prime})
  \rbrace
\end{align*}
of $Z(\overline{S})$ and we know that
\begin{align*}
  \gamma(1)
  &=
  \left(
    \mathrm{id}_{Z(S)}
    \otimes
    f^{-1}
  \right)
  \circ
  \mathrm{coev}_{Z(S)}^{\prime}(1)
  \\
  &=
  \left(
    \mathrm{id}_{Z(S)}
    \otimes
    f^{-1}
  \right)
  \left(
    \sum_{i = 1}^{k}
    b_{i}
    \otimes
    b_{i}^{\prime}
  \right)
  \\
  &=
  \sum_{i = 1}^{k}
  b_{i}
  \otimes
  f^{-1}(b_{i}^{\prime})
\end{align*}
We define
\begin{align*}
  \bar{b}_{i}
  &:=
  f^{-1}(b_{i}^{\prime})
  \qquad
  \text{for}
  \qquad
  i
  \in
  \lbrace
    1
    ,
    \ldots
    ,
    k
  \rbrace
\end{align*}
so that
\begin{align*}
  \Braket{\bar{b}_{i},b_{j}}
  &=
  f(\bar{b}_{i})(b_{j})
  \\
  &=
  b_{i}^{\prime}(b_{j})
  \\
  &=
  \delta_{ij}
\end{align*}
where $j \in \lbrace 1,\ldots,k\rbrace$.
\\
With this preparation we now prove a related property, namely how one can compute the dimension of the state space by taking the product with the circle $S^{1}$. It is the following
\\
\begin{cor}
\label{cor:dimS1}
Let
\begin{align*}
  Z
  \colon
  \mathbf{Cob}_{n}
  &\to
  \mathbf{Vec}_{K}
\end{align*}
be a TQFT and let $S \in \mathrm{ob}_{\mathbf{Cob}_{n}}$ then
\begin{align*}
  \Phi^{-1}
  \left(
    Z([S \times S^{1}])
    \left(
      \Phi(1)
    \right)
  \right)
  &=
  \dim(Z(S))
  =:
  k
  \in
  \mathbb{N}
\end{align*}
\end{cor}
\begin{prf}
The cobordism
\begin{align*}
  S
  \times
  S^{1}
  \colon
  \emptyset
  &\to
  \emptyset
\end{align*}
is a representative of the composition
\begin{align*}
  \mathrm{ev}_{S}
  \circ
  \mathsf{B}(S,\overline{S})
  \circ
  \mathrm{coev}_{S}
\end{align*}
for, as illustrated in figure \ref{fig:dimcirc}, we can simply untwist\footnote{remember that we are considering abstract manifolds} the braiding by turning around one of the bent cylinders representing $\mathrm{ev}_{S}$ or $\mathrm{coev}_{S}$.
\\
\begin{figure}[h!]
\centering
\begin{tikzpicture}[tqft/cobordism edge/.style={draw}]
  %left
  %coevaluation
  \pic[tqft,name=co,cobordism height=3cm,boundary separation=1.75cm,incoming boundary components=0,outgoing boundary components=2,every outgoing upper boundary component/.style={draw},every outgoing boundary component/.style={draw,ultra thin,dashed}];
  \node[at=(co-outgoing boundary 1),font=\small]{$S$};
  \node[at=(co-outgoing boundary 2),below=-7pt,font=\small]{$\overline{S}$};
  \node[at=(co-between outgoing 1 and 2),above=3pt,font=\small]{$\mathrm{coev}_{S}$};
  %braiding
  \pic[tqft/cylinder to next,name=l,boundary separation=3.5cm,at=(co-outgoing boundary 1)];
  \pic[tqft/cylinder to prior,name=r,boundary separation=3.5cm,at=(co-outgoing boundary 2)];
  \node[at=(r-between first incoming and first outgoing),left=1cm,font=\small]{$\mathsf{B}(S,\overline{S})$};
  %evaluation
  \pic[tqft,name=ev,cobordism height=3cm,boundary separation=1.75cm,incoming boundary components=2,outgoing boundary components=0,anchor=incoming boundary 1,at=(r-outgoing boundary 1),every incoming upper boundary component/.style={draw},every incoming lower boundary component/.style={draw,ultra thin,dashed}];
  \node[at=(ev-incoming boundary 1),below=-7pt,font=\small]{$\overline{S}$};
  \node[at=(ev-incoming boundary 2),font=\small]{$S$};
  \node[at=(ev-between incoming 1 and 2),below=4pt,font=\small]{$\mathrm{ev}_{S}$};

  %right
  \node[at=(r-between first incoming and first outgoing),right=2.8cm,font=\small]{$\cong$};
  %top
  \pic[tqft,name=co2,cobordism height=3cm,boundary separation=1.75cm,incoming boundary components=0,outgoing boundary components=2,anchor={(-1,0.65)},at=(co-outgoing boundary 2),outgoing upper boundary component 1/.style={draw},outgoing lower boundary component 1/.style={draw,ultra thin,dashed}];
  \node[at=(co2-outgoing boundary 1),font=\small]{$S$};
  %bottom
  \pic[tqft,name=ev2,cobordism height=3cm,boundary separation=1.75cm,incoming boundary components=2,outgoing boundary components=0,anchor=incoming boundary 1,at=(co2-outgoing boundary 1)];
  \node[at=(co2-outgoing boundary 2),right=2em,font=\small]{$S \times S^{1}$};
\end{tikzpicture}
\caption{Illustration of the product manifold with the circle}
\label{fig:dimcirc}
\end{figure}
\\
Using the coherence condition for $\mathsf{B}$ and $\mathsf{H}$ we find by applying $Z$ that
\begin{align*}
  &
  \Phi^{-1}
  \circ
  Z([S \times S^{1}])
  \circ
  \Phi
  \\
  &=
  \Phi^{-1}
  \circ
  Z(\mathrm{ev}_{S})
  \circ
  Z(\mathsf{B}(S,\overline{S}))
  \circ
  Z(\mathrm{coev}_{S})
  \circ
  \Phi
  \\
  &=
  \Phi^{-1}
  \circ
  Z(\mathrm{ev}_{S})
  \circ
  \mathsf{H}(\overline{S},S)
  \circ
  \mathsf{B}(Z(S),Z(\overline{S}))
  \circ
  \mathsf{H}^{-1}(S,\overline{S})
  \circ
  Z(\mathrm{coev}_{S})
  \circ
  \Phi
  \\
  &=
  \Braket{\cdot,\cdot}
  \circ
  \mathsf{B}(Z(S),Z(\overline{S}))
  \circ
  \gamma
\end{align*}
Hence, choosing bases for $Z(S)$ and $Z(\overline{S})$ as above, we have
\begin{align*}
  \Phi^{-1}
  \left(
    Z([S \times S^{1}])
    \left(
      \Phi(1)
    \right)
  \right)
  &=
  \Braket{\cdot,\cdot}
  \left(
    \mathsf{B}(Z(S),Z(\overline{S}))
    \left(
      \gamma(1)
    \right)
  \right)
  \\
  &=
  \Braket{\cdot,\cdot}
  \left(
    \mathsf{B}(Z(S),Z(\overline{S}))
    \left(
      \sum_{i = 1}^{k}
      b_{i}
      \otimes
      \bar{b}_{i}
    \right)
  \right)
  \\
  &=
  \sum_{i = 1}^{k}
  \Braket{\bar{b}_{i},b_{i}}
  \\
  &=
  k
  \\
  &=
  \dim(Z(S))
\end{align*}
finishing the proof.
\\
\phantom{proven}
\hfill
$\square$
\end{prf}
