%\nocite{f215dbd0}
%\nocite{cc6d78b5}
%\nocite{dfcdc48c}
%\nocite{8b5861fc}
%\nocite{00000001}
%\nocite{wiki-nlab0000}
%\nocite{wiki-pedia0en}
%
%
%%%
Whenever one aims to describe something it is important to have a good language or framework in which the description can be given. The right language can simplify the description and increase the understanding significantly and hence make for an improved reasoning. The development of a good mathematical framework in physics exemplifies this in an often amazing way. Coming up with new ideas for the mathematical language can give better insights in the physics to be described. In recent years it has come to light that category theory can provide a well-suited framework, especially in quantum field theory. While in general this is still in the beginnings, there is a special case for which quite some progress has been made, namely topological quantum field theories (TQFTs). These are interesting not only for physicists but also for mathematicians. Ordinary category theory is well-suited for describing ordinary TQFTs but one really aims to understand extended TQFTs and here higher categories come into play as a proper framework.\\
What we want to do in this work is to give an introduction to TQFTs, starting with ordinary TQFTs to get a feeling for what a TQFT looks like and then examine how one is led to extended TQFTs. As we want to address mathematicians and physicists alike, and in particular mathematical physicists, we moreover try to motivate the categorical definition of a TQFT from concepts which are more widespread in physics. Furthermore, our description of the ordinary category of cobordisms is rather thorough and we also try to make clear what higher cobordisms used for extended TQFTs are. This is often neglected a bit in the existing literature on the topic and should be helpful for the reader unfamiliar with these notions.
\\\\
Apart from some basic undergraduate math like linear algebra here and there it is necessary to know some basics of category theory - in particular categories, functors and natural transformations and moreover adjoints, limits and the idea of universality - in order to understand this text. Furthermore one should have a good idea of symmetric monoidal categories, but we will recall the definitions here and the reader unfamiliar with them possibly can get a basic understanding along the way.
\\
A good understanding of manifolds is also almost mandatory as we only give the basic definitions and results we need, which, of course, cannot replace a thorough introduction as given in some standard textbook. The same is true for the concept of fiber bundles, particularly principal bundles, vector bundles and their relation. Moreover, some knowledge about homotopy theory and also simplicial sets is very helpful, especially when it comes to higher categories. Yet, as our treatment of higher categories is rather informal, at least simplicial sets are not strictly needed for most of this work.
\\
A perfect reader is already familiar with the concept of higher categories since our treatment of them is not perfectly detailed (yet hopefully sufficient to be able to understand this text even without prior knowledge). Besides, such a perfect reader knows some singular homology, which is needed for parts of the concept of orientation, because we only give a short axiomatic description of homology in the appendix without providing intuition. Finally, a basic understanding of quantum field theory is helpful, at least from a physical point of view. We give a very rough overview of the main ideas to put this work in context and hopefully this is enough to be a good motivation to read on. To understand this motivation some basic knowledge about classical field theory and quantum mechanics is required. However, most of the text can be understood without any knowledge of quantum field theory and should still be interesting from a mathematical point of view.
\\
Much of these prerequisites is covered in \cite{00000001}, in particular the categorical parts. Just as done there we use Tarski-Grothendieck set theory as a foundation in order not to have to care about {\glqq}size issues{\grqq}. See \cite{00000001} for more on this. In fact, after reading this reference one should be sufficiently prepared to understand pretty much all of the present work. For the other concepts we refer to standard textbooks like \cite{8b5861fc} for basic homotopy theory\footnote{actually this is also sufficiently treated in \cite{00000001}} and homology or \cite{f215dbd0} for bundles. For quantum field theory from a mathematical perspective a good point to start is the nLab, \cite{wiki-nlab0000}, which also is a good first reference in general as is Wikipedia, \cite{wiki-pedia0en}. Every now and then there will be notions written in \textit{italic} style. These are notions which actually need a precise definition we do not provide because they are not central for the understanding or because we have given an informal description which is enough here. So, a reader who is unfamiliar with these notions should be able to read on even without a deeper understanding.
\\\\
Before starting with the main text we give an overview of quantum field theory in chapter \ref{chap:motqft} which should serve as a motivation from a physical point of view as already mentioned above.
\\
The main text is divided into two parts. The first part is about ordinary TQFTs. In chapter \ref{chap:prelim1} we start with recalling the necessary definitions for symmetric monoidal categories and dual objects. Afterwards we briefly discuss principal bundles and vector bundles and how they are related. Then we introduce the concept of orientation on manifolds for the reader unfamiliar with it. After these preliminaries we spend the subsequent chapter \ref{chap:defordtqft} on introducing the categories involved in the definition of a TQFT, the cobordism category and the category of vector spaces. We finish chapter \ref{chap:defordtqft} with finally defining ordinary TQFTs. As a short interlude in chapter \ref{chap:motpathint} we try to motivate this categorical definition from properties of the path integral which is widely used in quantum field theory in physics. In the final chapter \ref{CHAP:ALTCHARPROPS} of this part we are concerned with some basic statements about ordinary TQFTs. We start with giving an alternative characterization and then examine how to compare TQFTs. Afterwards we consider some basic general properties of TQFTs. Many of these things easily follow from purely categorical considerations.
\\
The second part is about extended TQFTs and the cobordism hypothesis which was originally stated by Baez and Dolan in \cite{cc6d78b5}. This part is essentially based on Lurie's expository paper \cite{dfcdc48c} about a precise formulation and a proof of the cobordism hypothesis. We start with some preliminaries in chapter \ref{chap:prelim2}, more precisely we give an informal introduction to higher categories and we introduce manifolds with corners and extended cobordisms. In chapter \ref{chap:lowdimtqft} we consider ordinary TQFTs of low dimension as a motivation for extended TQFTs. The following chapter \ref{chap:extcob} then is about extending the ordinary category of cobordisms to obtain a more sophisticated higher category of cobordisms. This is again done on an informal level as our treatment about higher categories was already informal. In the final chapter \ref{chap:formcobhyp} of this part we give a formulation of several versions of the cobordism hypothesis in the framework developed in the preceding chapters. We do not attempt for a description of the proof here as this is rather elaborated and we have neither the space and time nor the means to do this.
\\
After the main text we round off this work with a brief outlook in chapter \ref{chap:outlook}.
\\
The appendix includes a brief axiomatic introduction to homology, a sketch of the main idea of morse theory and moreover some of the tedious and little insightful parts of some proofs from chapter \ref{CHAP:ALTCHARPROPS} in the main part.
\\\\
We try to give a good intuition for the central ideas and still provide enough formal details to get a thorough understanding of the important parts. In some places we are more formal, for example in the orientation part especially when dealing with homology, but in other places, in particular when it comes to higher categories, we are quite informal, only trying to make clear what the idea is.
\\
In order to improve readability and comprehensibility some definitions are repeated here and there which, of course, comes at the cost of increased length. Moreover, the text is written in a rather decompressed way which considerably contributes to its length but hopefully makes it much easier to read.
\\
Definitions are given in the running text in \textbf{boldface} by phrases like {\glqq}formula such-and-such is called \textbf{name}{\grqq} or {\glqq}formula 1 is \textbf{name} if formula 2{\grqq}. Sometimes parts of the boldface text are written in parentheses which indicates that these parts are often left implicit in practice.
\\
Usually, statements are given in separate environments like lemma, theorem and corollary but minor statements are sometimes also given in the running text. If the statement is given in a separate environment then so is its proof which is closed with a $\Box$. For the statements in the running text the arguments for their proofs are also given in the running text. Sometimes proofs are only sketched or omitted altogether but usually there is a reference then where a more detailed proof can be found.
\\
Finally, there are several examples, often serving as illustrations but sometimes also as important parts of other statements. Mostly, examples are given in separate environments, too, on the one hand for later reference but on the other hand also for further structuring the text.

