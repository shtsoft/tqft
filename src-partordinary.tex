This first part is about ordinary TQFTs as first brought up in (almost) the axiomatization considered here by Atiyah in 1988. The description is given in the framework of ordinary categories or more precisely symmetric monoidal categories.
\\\\
The part is divided into four chapters starting with some preliminaries in chapter \ref{chap:prelim1} which are needed in the subsequent chapters of this and partly also of the second part. The next chapter, chapter \ref{chap:defordtqft}, is about the definition of TQFTs in categorical terms. In the following chapter \ref{chap:motpathint} we want to motivate this categorical definition from a formulation of QFT which is more widely used among physicists, the path integral. In the final chapter \ref{CHAP:ALTCHARPROPS} of this part we then consider some basic statements which hold true for any ordinary TQFT.
\\\\
The objective of this part is to give an introduction to ordinary TQFTs and convey some intuition for them. So, the reader who does not know about the subject should become familiar with ordinary TQFTs in the course of this part.
