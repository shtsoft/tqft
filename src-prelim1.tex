%\nocite{00000001}
%%%
Let us start with the preliminaries of this part. In section \ref{sec:smoncatdual} we first recall the definitions of symmetric monoidal categories and dual objects. We do not give intuition here but the careful reader might notice that (symmetric) monoidal categories are a categorification of (commutative) monoids. The unexperienced reader may have a look at \cite{00000001} for more on this. Then in section \ref{sec:bundles} we briefly discuss some basics of bundles, particularly principal bundles, vector bundles and how they are related. The final section \ref{sec:orient} of this chapter is dedicated to the concept of orientation of manifolds. This is needed for one of the categories discussed in the definition of ordinary TQFTs, namely the category of cobordisms. We will usually deal with smooth manifolds, yet as we will also touch on oriented topological manifolds, and maybe for a better understanding of the concept altogether, we also consider orientability for topological manifolds. To properly understand orientation in this more general setting we need some basics of (singular) homology. The latter is briefly introduced in the appendix, though without giving proofs or intuition.
