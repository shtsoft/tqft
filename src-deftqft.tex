\nocite{0a816f4c}
%%%
Now it is easy to state the definition of a TQFT: for $n \in \mathbb{N}^{\times}$ an \textbf{(ordinary\footnote{in this part we only consider ordinary TQFTs, so we usually drop this adjective})} $n$-\textbf{dimensional (oriented) topological quantum field theory} or short $n$-\textbf{dimensional TQFT} is defined to be a symmetric monoidal functor
\begin{align*}
  Z
  &\doteq
  (Z,\mathsf{H},\Phi)
  \colon
  \mathbf{Cob}_{n}
  \to
  \mathbf{Vec}_{K}
\end{align*}
Thus an $n$-dimensional TQFT assigns to every closed $(n-1)$-dimensional manifold $S \in \mathrm{ob}_{\mathbf{Cob}_{n}}$ a vector space $Z(S)$ and to every cobordism class $[M]$ from $S_{1}$ to $S_{2}$ a linear map from $Z(S_{1})$ to $Z(S_{2})$. Moreover, disjoint unions of $(n-1)$-manifolds are basically taken to tensor products of vector spaces and disjoint unions of cobordism classes are basically taken to tensor products of linear maps in a compatible way and the empty $(n-1)$-manifold $\emptyset$ is basically taken to the field $K$. Finally, interchanging factors in a disjoint union basically translates to interchanging the corresponding factors in the assigned tensor product.
