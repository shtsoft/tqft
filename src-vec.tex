\nocite{0a816f4c}
%%%
The second category involved in the definition of a TQFT is $\mathbf{Vec}_{K}$ whose objects are the vector spaces over the field\footnote{we can usually think of the field as $\mathbb{R}$ or $\mathbb{C}$} $K$ and whose morphisms are linear maps between these vector spaces. It is fairly obvious that this is indeed a category, but it also carries a symmetric monoidal structure not coming from the (co)product\footnote{in this category the (binary) product and the (binary) coproduct are both given by the direct sum because for finitely many summands/factors the direct sum and the direct product coincide} which we briefly want to describe in the following.
\\
The tensor product of this symmetric monoidal structure is given by the usual tensor product of vector spaces $\otimes$. Recall that this can be constructed by using the free vector space modulo an appropriate equivalence relation. More precisely, let $F(B)$ denote the free vector space over $K$ over some set $B$, i.e. the formal finite linear combinations with the usual vector space structure. Then for vector spaces $V_{1},V_{2} \in \mathrm{ob}_{\mathbf{Vec}_{K}}$ their tensor product is
\begin{align*}
  V_{1}
  \otimes
  V_{2}
  &:=
  F(V_{1} \times V_{2})
  /
  \sim
\end{align*}
where $\sim$ is the equivalence relation generated by
\begin{align*}
  &
  (v_{1},v_{2})
  +
  (\hat{v}_{1},v_{2})
  \sim
  (v_{1} + \hat{v}_{1},v_{2})
  \\
  &
  (v_{1},v_{2})
  +
  (v_{1},\hat{v}_{2})
  \sim
  (v_{1},v_{2} + \hat{v}_{2})
  \\
  &
  (cv_{1},v_{2})
  \sim
  c
  (v_{1},v_{2})
  \sim
  (v_{1},cv_{2})
\end{align*}
for $v_{1},\hat{v}_{1} \in V_{1}$, $v_{2},\hat{v}_{2} \in V_{2}$ and $c \in K$. The vector space structure is then given by doing the vector space operations on representatives and then taking the equivalence class. One easily checks that this is well-defined. For $v_{1} \in V_{1}$ and $v_{2} \in V_{2}$ the corresponding equivalence class in $V_{1} \otimes V_{2}$ is denoted $v_{1} \otimes v_{2}$ and such a vector is called a pure tensor. Note that by construction every vector $a \in V_{1} \otimes V_{2}$ can be written as a finite linear combination of pure tensors, that is
\begin{align*}
  a
  &=
  \sum_{j=1}^{m}
  v_{1}^{j}
  \otimes
  v_{2}^{j}
\end{align*}
for some $m \in \mathbb{N}^{\times}$ and $v_{1}^{j} \in V_{1}$, $v_{2}^{j} \in V_{2}$, $1 \leq j \leq m$. Note further that
\begin{align*}
  &
  v_{1}
  \otimes
  v_{2}
  +
  \hat{v}_{1}
  \otimes
  v_{2}
  =
  (v_{1} + \hat{v}_{1})
  \otimes
  v_{2}
  \\
  &
  v_{1}
  \otimes
  v_{2}
  +
  v_{1}
  \otimes
  \hat{v}_{2}
  =
  v_{1}
  \otimes
  (v_{2} + \hat{v}_{2})
  \\
  &
  (cv_{1})
  \otimes
  v_{2}
  =
  c
  (v_{1} \otimes v_{2})
  =
  v_{1}
  \otimes
  (cv_{2})
\end{align*}
reflecting the equivalence relation and we will sometimes refer to these properties as the {\glqq}bilinearity of the tensor product{\grqq}. This also implies that for finite-dimensional vector spaces $V_{1}$ and $V_{2}$ of dimension $m_{1}$ and $m_{2}$ and bases
\begin{align*}
  \lbrace
    v_{1}^{1}
    ,
    \ldots
    ,
    v_{1}^{m_{1}}
  \rbrace
  \qquad
  \text{and}
  \qquad
  \lbrace
    v_{2}^{1}
    ,
    \ldots
    ,
    v_{2}^{m_{2}}
  \rbrace
\end{align*}
respectively, the tensor product $V_{1} \otimes V_{2}$ has dimension $m_{1} \cdot m_{2}$ and a basis is given by
\begin{align*}
  \lbrace
    v_{1}^{j_{1}}
    \otimes
    v_{2}^{j_{2}}
    \colon
    1
    \leq
    j_{1}
    \leq
    m_{1}
    \ 
    \land
    \ 
    1
    \leq
    j_{2}
    \leq
    m_{2}
  \rbrace
\end{align*}
There is a universal property characterizing the tensor product of vector spaces (up to isomorphism as always). More precisely, for vector spaces $V_{1},V_{2}$ let
\begin{align*}
  \mathrm{Bilin}(V_{1},V_{2};-)
  \colon
  \mathbf{Vec}_{K}
  &\to
  \mathbf{Set}
\end{align*}
be the functor which sends a vector space $W$ to the set of bilinear\footnote{remember that bilinear means linear in each argument separately} maps
\begin{align*}
  \mathrm{Bilin}(V_{1},V_{2};W)
\end{align*}
from $V_{1} \times V_{2}$ to $W$ and which takes a linear map
\begin{align*}
  L
  \in
  \mathrm{mor}_{\mathbf{Vec}_{K}}
  \left(
    W
    ,
    \hat{W}
  \right)
\end{align*}
to the function which composes with $L$, i.e.
\begin{align*}
  \mathrm{Bilin}(V_{1},V_{2};L)
  \colon
  \mathrm{Bilin}(V_{1},V_{2};W)
  &\to
  \mathrm{Bilin}(V_{1},V_{2};\hat{W})
  ,\qquad
  \phi
  \mapsto
  L
  \circ
  \phi
\end{align*}
Now for the tensor product we have the associated bilinear map
\begin{align*}
  \varphi_{V_{1},V_{2}}
  \colon
  V_{1}
  \times
  V_{2}
  &\to
  V_{1}
  \otimes
  V_{2}
  ,\qquad
  (v_{1},v_{2})
  \mapsto
  v_{1}
  \otimes
  v_{2}
\end{align*}
and this is an initial object in the category of elements of $\mathrm{Bilin}(V_{1},V_{2};-)$
\begin{align*}
  \int_{\mathbf{Vec}_{K}}
  \mathrm{Bilin}(V_{1},V_{2};-)
\end{align*}
For a more explicit description note that this category can be viewed as having as objects the bilinear maps
\begin{align*}
  \phi
  \colon
  V_{1}
  \times
  V_{2}
  &\to
  W
\end{align*}
to some vector space $W$ and having as morphisms the linear maps $L \colon W \to \hat{W}$ such that
\begin{equation*}
\begin{tikzcd}[row sep=large,column sep=4.3em]
  &
  W
  \ar{ld}[swap]{L}
  &
  \\
  \hat{W}
  &
  &
  V_{1} \times V_{2}
  \ar{lu}[swap]{\phi}
  \ar{ll}{\hat{\phi}}
\end{tikzcd}
\end{equation*}
commutes where
\begin{align*}
  \hat{\phi}
  \colon
  V_{1}
  \times
  V_{2}
  &\to
  \hat{W}
\end{align*}
is another object. Then for $\varphi_{V_{1},V_{2}}$ to be an initial object means that for any bilinear map
\begin{align*}
  \phi
  \colon
  V_{1}
  \times
  V_{2}
  &\to
  W
\end{align*}
to any vector space $W$ there is a unique linear map
\begin{align*}
  L_{!}
  \colon
  V_{1}
  \otimes
  V_{2}
  &\to
  W
\end{align*}
making the follwing diagram commute
\begin{equation*}
\begin{tikzcd}[row sep=large,column sep=4.3em]
  &
  V_{1} \otimes V_{2}
  \ar{ld}[swap]{L_{!}}
  &
  \\
  W
  &
  &
  V_{1} \times V_{2}
  \ar{lu}[swap]{\varphi_{V_{1},V_{2}}}
  \ar{ll}{\phi}
\end{tikzcd}
\end{equation*}
For more on this universal property the interested reader is referred to \cite{52fbba46}.
\\
Now for two linear maps
\begin{align*}
  L
  \in
  \mathrm{mor}_{\mathbf{Vec}_{K}}
  \left(
    V_{1}
    ,
    V_{2}
  \right)
  ,\qquad
  \tilde{L}
  \in
  \mathrm{mor}_{\mathbf{Vec}_{K}}
  \left(
    \tilde{V}_{1}
    ,
    \tilde{V}_{2}
  \right)
\end{align*}
their tensor product is defined by
\begin{align*}
  L
  \otimes
  \tilde{L}
  \colon
  V_{1}
  \otimes
  \tilde{V}_{1}
  &\to
  V_{2}
  \otimes
  \tilde{V}_{2}
  ,\qquad
  v_{1}
  \otimes
  \tilde{v}_{1}
  \mapsto
  L(v_{1})
  \otimes
  \tilde{L}(\tilde{v}_{1})
\end{align*}
on pure tensors and the demand for linearity. With this construction it is easily checked that the tensor product is a functor
\begin{align*}
  \otimes
  \colon
  \mathbf{Vec}_{K}
  \times
  \mathbf{Vec}_{K}
  &\to
  \mathbf{Vec}_{K}
\end{align*}
The unit object for the tensor product is given by the field
\begin{align*}
  K = 1_{\mathbf{Vec}_{K}}
\end{align*}
From the universal property of the tensor product one can construct the symmetric monoidal structure of $\mathbf{Vec}_{K}$. But we can easily give the natural isomorphisms more explicitly. For $V_{1},V_{2},V_{3} \in \mathrm{ob}_{\mathbf{Vec}_{K}}$ we define
\begin{align*}
  \mathsf{A}(V_{1},V_{2},V_{3})
  \colon
  \left(
    (V_{1} \otimes V_{2})
    \otimes
    V_{3}
  \right)
  &\to
  V_{1}
  \otimes
  (V_{2} \otimes V_{3})
  \\
  \mathsf{A}(V_{1},V_{2},V_{3})
  \left(
    (v_{1} \otimes v_{2})
    \otimes
    v_{3}
  \right)
  &:=
  v_{1}
  \otimes
  (v_{2} \otimes v_{3})
  \\
  \mathsf{L}(V)
  \colon
  K
  \otimes
  V
  &\to
  V
  \\
  \mathsf{L}(V)(1 \otimes v)
  &:=
  v
  \\
  \mathsf{R}(V)
  \colon
  V
  \otimes
  K
  &\to
  V
  \\
  \mathsf{R}(V)(v \otimes 1)
  &:=
  v
  \\
  \mathsf{B}(V_{1},V_{2})
  \colon
  V_{1}
  \otimes
  V_{2}
  &\to
  V_{2}
  \otimes
  V_{1}
  \\
  \mathsf{B}(V_{1},V_{2})(v_{1} \otimes v_{2})
  &:=
  v_{2}
  \otimes
  v_{1}
\end{align*}
on pure tensors and expand as linear maps where necessary. Here $1$ denotes the multiplicative identity in $K$. Note that every element in $K \otimes V$ can be written as $1 \otimes v$ with $v \in V$ and likewise for $V \otimes K$. The naturality of $\mathsf{A}$, $\mathsf{L}$, $\mathsf{R}$ and $\mathsf{B}$ are easily verified and so are the pentagon, triangle and hexagon equations but we will not go into details here. One simply checks things on pure tensors and uses that everything is linear. It is also evident that the braiding is symmetric.
\\\\
Finally we consider dual objects in $\mathbf{Vec}_{K}$. We will see that there are dual objects but not for all objects.
\\
\begin{lem}
\label{lem:dualvec}
For a finite-dimensional vector space $V \in \mathrm{ob}_{\mathbf{Vec}_{K}}$ of dimension $k \in \mathbb{N}$ a dual object in $\mathbf{Vec}_{K}$ is given by the dual vector space $V^{\prime}$, i.e. the vector space of all linear maps from $V$ to $K$. Choosing a basis
\begin{align*}
  \lbrace
    b_{1}
    ,
    \ldots
    ,
    b_{k}
  \rbrace
\end{align*}
for $V$ we obtain a dual basis
\begin{align*}
  \lbrace
    b_{1}^{\prime}
    ,
    \ldots
    ,
    b_{k}^{\prime}
  \rbrace
\end{align*}
for $V^{\prime}$, that is, a basis with
\begin{align*}
  b_{i}^{\prime}(b_{j})
  &=
  \delta_{ij}
  \qquad
  \text{for }
  1
  \leq
  i,j
  \leq
  k
\end{align*}
where $\delta_{ij}$ is the Kronecker delta. The evaluation and coevaluation are then given by
\begin{align*}
  \mathrm{ev}_{V}
  \colon
  V^{\prime}
  \otimes
  V
  \to
  K
  &,\qquad
  v^{\prime}
  \otimes
  v
  \mapsto
  v^{\prime}(v)
  \\
  \mathrm{coev}_{V}
  \colon
  K
  \to
  V
  \otimes
  V^{\prime}
  &,\qquad
  \mathrm{coev}_{V}(1)
  :=
  \sum_{i = 1}^{k}
  b_{i}
  \otimes
  b_{i}^{\prime}
\end{align*}
and the demand for linearity.
\\
If the vector space is inifnite-dimensional then there is no dual object in $\mathbf{Vec}_{K}$.
\end{lem}
\begin{prf}
\begin{enumerate}
\item[(i)]
To see that the requested diagrams (LD1) and (LD2) for dual objects commute in the case of a finite-dimensional vector space $V$ let $u \in V$. We can write
\begin{align*}
  u
  &=
  \sum_{i=1}^{k}
  u^{i}
  b_{i}
\end{align*}
for certain $u^{i} \in K$ and applying the $b_{j}^{\prime}$ yields
\begin{align*}
  b_{j}^{\prime}(u)
  &=
  u^{j}
  ,\qquad
  \text{for }
  1
  \leq
  j
  \leq k
\end{align*}
and hence
\begin{align*}
  u
  &=
  \sum_{i=1}^{k}
  b_{i}^{\prime}(u)
  b_{i}
\end{align*}
Using that
\begin{align*}
  \mathsf{L}^{-1}(V)(u)
  &=
  1
  \otimes
  u
\end{align*}
we can calculate
\begin{align}
\label{vecld1}
\begin{split}
  &
  \mathsf{R}(V)
  \circ
  \left(
    \mathrm{id}_{V}
    \otimes
    \mathrm{ev}_{V}
  \right)
  \circ
  \mathsf{A}(V,V^{\prime},V)
  \circ
  \left(
    \mathrm{coev}_{V}
    \otimes
    \mathrm{id}_{V}
  \right)
  \circ
  \mathsf{L}^{-1}(V)(u)
  \\
  &=
  \mathsf{R}(V)
  \circ
  \left(
    \mathrm{id}_{V}
    \otimes
    \mathrm{ev}_{V}
  \right)
  \circ
  \mathsf{A}(V,V^{\prime},V)
  \left(
    \sum_{i = 1}^{k}
    \left(
      b_{i}
      \otimes
      b_{i}^{\prime}
    \right)
    \otimes
    u
  \right)
  \\
  &=
  \mathsf{R}(V)
  \circ
  \left(
    \mathrm{id}_{V}
    \otimes
    \mathrm{ev}_{V}
  \right)
  \left(
    \sum_{i = 1}^{k}
    b_{i}
    \otimes
    \left(
      b_{i}^{\prime}
      \otimes
      u
    \right)
  \right)
  \\
  &=
  \mathsf{R}(V)
  \left(
    \sum_{i = 1}^{k}
    b_{i}
    \otimes
    b_{i}^{\prime}(u)
  \right)
  \\
  &=
  \mathsf{R}(V)
  \left(
    \sum_{i = 1}^{k}
    b_{i}^{\prime}(u)
    b_{i}
    \otimes
    1
  \right)
  \\
  &=
  \sum_{i = 1}^{k}
  b_{i}^{\prime}(u)
  b_{i}
  \\
  &=
  u
\end{split}
\end{align}
Here we made use of the bilinearity of the tensor product and the linearity of the involved maps. This shows that the first diagram commutes and the other diagram can be treated similarly.

\item[(ii)]
To see that the finite-dimensional vector spaces are the only ones that have dual objects let $V$ be any vector space and suppose $V^{\ast}$ is a dual object with evaluation and coevaluation $\mathrm{ev}_{V}$ and $\mathrm{coev}_{V}$. Now, there are $m \in \mathbb{N}$ and $b_{i} \in V$ and $b_{i}^{\ast} \in V^{\ast}$ for $1 \leq i \leq m$, such that
\begin{align*}
  \mathrm{coev}_{V}(1)
  &=
  \sum_{i = 1}^{m}
  b_{i}
  \otimes
  b_{i}^{\ast}
\end{align*}
since every element of $V \otimes V^{\ast}$ is of this form. With the same calculation as in equation \eqref{vecld1} in the finite-dimensional case in a slightly different order we find from the first diagram governing $\mathrm{ev}_{V}$ and $\mathrm{coev}_{V}$ that for $u \in V$ we have
\begin{align*}
  u
  &=
  \mathrm{id}_{V}(u)
  \\
  &=
  \mathsf{R}(V)
  \circ
  \left(
    \mathrm{id}_{V}
    \otimes
    \mathrm{ev}_{V}
  \right)
  \circ
  \mathsf{A}(V,V^{\ast},V)
  \circ
  \left(
    \mathrm{coev}_{V}
    \otimes
    \mathrm{id}_{V}
  \right)
  \circ
  \mathsf{L}^{-1}(V)(u)
  \\
  &=
  \sum_{i = 1}^{m}
  \mathrm{ev}_{V}(b_{i}^{\ast} \otimes u)
  b_{i}
\end{align*}
which shows that
\begin{align*}
  \mathrm{span}
  \left(
    \lbrace
      b_{1}
      ,
      \ldots
      ,
      b_{m}
    \rbrace
  \right)
  &=
  V
\end{align*}
and thus $\dim(V) \leq m$.
\end{enumerate}
\phantom{proven}
\hfill
$\Box$
\end{prf}
This concludes our treatment of the category of vector spaces.
