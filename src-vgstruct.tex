An interesting class of tangential structures arises when given a topological group $G$ and a continuous homomorphism $\chi \colon G \to O(n)$. Then we have a $G$-action induced by $\chi$ on every space with an $O(n)$-action by letting $g \in G$ act as $\chi(g) \in O(n)$. Now let $\mathrm{B}G$ be a classifying space of $G$ and $\mathrm{E}G$ the weakly contractible total space of the corresponding universal bundle $\hat{\xi}_{G}$. We consider the bundle $\zeta_{\chi} = \hat{\xi}_{G}[\mathbb{R}^{n}]_{\chi}$ over $\mathrm{B}G$ with fiber $\mathbb{R}^{n}$ associated with $\hat{\xi}_{G}$ determined by $\chi$. Remember that this means that the total space of $\zeta_{\chi}$ is
\begin{align*}
  E_{\zeta_{\chi}}
  &=
  \mathrm{E}G
  \times_{G,\chi}
  \mathbb{R}^{n}
  =
  (\mathrm{E}G \times \mathbb{R}^{n})/G
\end{align*}
where $G$ acts on $\mathrm{E}G \times \mathbb{R}^{n}$ by
\begin{align*}
  (p,x)g
  :=
  (pg,\chi(g^{-1})x)
\end{align*}
for $g \in G$ and $(p,x) \in \mathrm{E}G \times \mathbb{R}^{n}$. The bundle $\zeta_{\chi}$ is a real $n$-dimensional vector bundle over $\mathrm{B}G$, hence we can define a $(\mathrm{B}G,\zeta_{\chi})$-structure on a $k$-dimensional manifold with corners, $k \leq n$, and we call such a structure a \textbf{$G$-structure (on $M$)}. Moreover, we write $\mathbf{Bord}_{n}^{G}$ for the corresponding symmetric monoidal $(\infty,n)$-category $\mathbf{Bord}_{n}^{(\mathrm{B}G,\zeta_{\chi})}$.
\\
We want to sketch that a $G$-structure on $M$ is basically a change of the structure group of the $n$-stabilized tangent bundle of $M$ from $O(n)$ to $G$ along $\chi$. Remember that for an $n$-dimensional vector bundle with metric $\zeta$ - whose structure group is $O(n)$ - a reduction of the structure group to some topological group $G$ along a continuous homomorphism $\chi \colon G \to O(n)$ is a reduction of the structure group of the orthonormal frame bundle $\mathsf{OF}(\zeta)$. Remember further that this means that there exists a principal $G$-bundle $\xi$ over $B_{\zeta}$ and an isomorphism from $\mathsf{OF}(\zeta)$ to the principal $O(n)$-bundle $\xi[O(n)]_{\chi}$ over $B_{\zeta}$ with fiber $O(n)$ associated with $\xi$ determined by $\chi$, i.e. whose total space is $E_{\xi} \times_{G,\chi} O(n)$.
\\
First we observe
\\
\begin{lem}
\label{lem:redstrgrofzeta}
Let $G$ be a topological group and $\chi \colon G \to O(n)$ a continuous homomorphism. Let further $\hat{\xi}_{G}$ be the universal bundle of $G$ and $\zeta_{\chi} = \hat{\xi}_{G}[\mathbb{R}^{n}]_{\chi}$ the associated real $n$-dimensional vector bundle over $\mathrm{B}G$ associated with $\hat{\xi}_{G}$ determined by $\chi$. Then the orthonormal frame bundle $\mathsf{OF}(\zeta_{\chi})$ of $\zeta_{\chi}$ is isomorphic to the principal $O(n)$-bundle $\hat{\xi}_{G}[O(n)]_{\chi}$ over $\mathrm{B}G$ with fiber $O(n)$ associated with $\hat{\xi}_{G}$ determined by $\chi$.
\end{lem}
\begin{prf}[Sketch]
Remember that there is a bijection between isomorphism classes of principal $O(n)$-bundles over $\mathrm{B}G$ and isomorphism classes of real $n$-dimensional vector bundles over $\mathrm{B}G$ since the latter is paracompact as CW-complex. This bijection is given by taking a principal $O(n)$-bundle $\xi$ to its associated real $n$-dimensional vector bundle $\xi[\mathbb{R}^{n}]$ and in the other direction taking a real $n$-dimensional vector bundle $\zeta$ to its orthonormal frame bundle $\mathsf{OF}(\zeta)$. Thus it is enough to show that the associated vector bundles of the two $O(n)$-principal bundles $\mathsf{OF}(\zeta_{\chi})$ and $\hat{\xi}_{G}[O(n)]_{\chi}$ are isomorphic.
\\
Since
\begin{align*}
  \mathsf{OF}(\zeta_{\chi})[\mathbb{R}^{n}]
  &\cong
  \zeta_{\chi}
\end{align*}
we have to show that
\begin{align*}
  \zeta_{\chi}
  &\cong
  \hat{\xi}_{G}[O(n)]_{\chi}[\mathbb{R}^{n}]
\end{align*}
But this follows because their total spaces are isomorphic,
\begin{align*}
  E_{\zeta_{\chi}}
  &=
  \mathrm{E}G
  \times_{G,\chi}
  \mathbb{R}^{n}
  \\
  &\cong
  \mathrm{E}G
  \times_{G,\chi}
  \left(
    O(n)
    \times_{O(n)}
    \mathbb{R}^{n}
  \right)
  \\
  &\cong
  \left(
    \mathrm{E}G
    \times_{G,\chi}
    O(n)
  \right)
  \times_{O(n)}
  \mathbb{R}^{n}
  \\
  &=
  E_{\hat{\xi}_{G}[O(n)]_{\chi}[\mathbb{R}^{n}]}
\end{align*}
as the total space of $\hat{\xi}_{G}[O(n)]_{\chi}$ is $\mathrm{E}G \times_{G,\chi} O(n)$, and because these isomorphisms respect the projections of the bundles.
\\
\phantom{proven}
\hfill
$\Box$
\end{prf}
This shows that the structure group $O(n)$ of $\mathsf{OF}(\zeta_{\chi})$ can be changed to $G$ along $\chi$ which is of course not very surprising when contemplating the definition of $\zeta_{\chi}$. Furthermore, we have
\\
\begin{lem}
\label{lem:redstrgrofpbzeta}
Let $G$ a topological group and $\chi \colon G \to O(n)$ a continuous homomorphism. Let further $X$ be a paracompact Hausdorff space and $f \colon X \to \mathrm{B}G$ a continuous map. Then 
\begin{align*}
  \mathsf{OF}(f^{\ast}\zeta_{\chi})
  &\cong
  \left(
    f^{\ast}\hat{\xi}_{G}
  \right)
  [O(n)]_{\chi}
\end{align*}
\end{lem}
\begin{prf}
Since taking the associated bundle commutes with taking the pullback bundle, lemma \ref{lem:redstrgrofzeta} implies
\begin{align*}
  f^{\ast}
  \mathsf{OF}(\zeta_{\chi})
  &\cong
  f^{\ast}
  \left(
    \hat{\xi}_{G}[O(n)]_{\chi}
  \right)
  \\
  &\cong
  \left(
    f^{\ast}\hat{\xi}_{G}
  \right)
  [O(n)]_{\chi}
\end{align*}
But $f^{\ast}\mathsf{OF}(\zeta_{\chi})$ is isomorphic to the orthonormal frame bundle $\mathsf{OF}(f^{\ast}\zeta_{\chi})$ of the pullback of $\zeta_{\chi}$ as both are principal $O(n)$-bundles over a paracompact Hausdorff space and the corresponding associated vector bundles are isomorphic,
\begin{align*}
  \left(
    f^{\ast}
    \mathsf{OF}(\zeta_{\chi})
  \right)
  [\mathbb{R}^{n}]
  &\cong
  f^{\ast}
  \left(
    \mathsf{OF}(\zeta_{\chi})[\mathbb{R}^{n}]
  \right)
  \\
  &\cong
  f^{\ast}\zeta_{\chi}
  \\
  &\cong
  \mathsf{OF}(f^{\ast}\zeta_{\chi})[\mathbb{R}^{n}]  
\end{align*}
Hence we have
\begin{align*}
  \mathsf{OF}(f^{\ast}\zeta_{\chi})
  &\cong
  \left(
    f^{\ast}\hat{\xi}_{G}
  \right)
  [O(n)]_{\chi}
\end{align*}
which we had to show.
\\
\phantom{proven}
\hfill
$\Box$
\end{prf}
This shows that the structure group $O(n)$ of $\mathsf{OF}(f^{\ast}\zeta_{\chi})$ can be changed to $G$ along $\chi$ because $f^{\ast}\hat{\xi}_{G}$ is a principal $G$-bundle over $X$. Hence we find
\\
\begin{cor}
\label{cor:redstrgrsttbundle}
Let $G$ a topological group and $\chi \colon G \to O(n)$ a continuous homomorphism. Let further $(f,\phi)$ be a $G$-structure on a $k$-dimensional manifold $M$, $k \leq n$. Then the structure group of the $n$-stabilized tangent bundle $TM \oplus (M \times \mathbb{R}^{n-k})$ can be changed to $G$ along $\chi$.
\end{cor}
\begin{prf}
The isomorphism
\begin{align*}
  \phi
  \colon 
  TM
  \oplus
  (M \times \mathbb{R}^{n-k})
  &\to
  f^{\ast}\zeta_{\chi}
\end{align*}
of the $G$-structure on $M$ induces an isomorphism
\begin{align*}
  \mathsf{OF}
  \left(
    TM
    \oplus
    (M \times \mathbb{R}^{n-k})
  \right)
  &\cong
  \mathsf{OF}
  \left(
    f^{\ast}\zeta_{\chi}
  \right)
\end{align*}
so that lemma \ref{lem:redstrgrofpbzeta} implies\footnote{note that any manifold is a paracompact Hausdorff space}
\begin{align*}
  \mathsf{OF}
  \left(
    TM
    \oplus
    (M \times \mathbb{R}^{n-k})
  \right)
  &\cong
  \left(
    f^{\ast}\hat{\xi}_{G}
  \right)
  [O(n)]_{\chi}
\end{align*}
which is what we want.
\\
\phantom{proven}
\hfill
$\Box$
\end{prf}
Conversely we have
\\
\begin{cor}
\label{cor:gstructfromred}
Let $G$ a topological group and $\chi \colon G \to O(n)$ a continuous homomorphism. Let further $M$ be a $k$-dimensional manifold $M$, $k \leq n$, and $\tilde{\phi}$ an isomorphism changing the structure group of the $n$-stabilized tangent bundle $TM \oplus (M \times \mathbb{R}^{n-k})$ to $G$ along $\chi$. Then there is a continuous map $f \colon M \to \mathrm{B}G$, unique up to homotopy, and an isomorphism
\begin{align*}
  TM
  \oplus
  (M \times \mathbb{R}^{n-k})
  &\cong
  f^{\ast}\zeta_{\chi}
\end{align*}
\end{cor}
\begin{prf}
Let $\xi$ be the principal $G$-bundle over $M$ of the reduction of the structure group, that is,
\begin{align*}
  \tilde{\phi}
  \colon
  \mathsf{OF}
  \left(
    TM
    \oplus
    (M \times \mathbb{R}^{n-k})
  \right)
  &\to
  \xi[O(n)]_{\chi}
\end{align*}
Due to the universal\footnote{note that a smooth manifold can always be endowed with the structure of a CW-complex} property of the universal bundle there is a continuous map $f \colon M \to \mathrm{B}G$, unique up to homotopy, such that
\begin{align*}
  \xi
  &\cong
  f^{\ast}\hat{\xi}_{G}
\end{align*}
Hence, lemma \ref{lem:redstrgrofpbzeta} implies
\begin{align*}
  \mathsf{OF}
  \left(
    TM
    \oplus
    (M \times \mathbb{R}^{n-k})
  \right)
  &\cong
  \left(
    f^{\ast}\hat{\xi}_{G}
  \right)
  [O(n)]_{\chi}
  \\
  &\cong
  \mathsf{OF}
  \left(
    f^{\ast}\zeta_{\chi}
  \right)
\end{align*}
and thus
\begin{align*}
  TM
  \oplus
  (M \times \mathbb{R}^{n-k})
  &\cong
  f^{\ast}\zeta_{\chi}
\end{align*}
which we had to show.
\\
\phantom{proven}
\hfill
$\Box$
\end{prf}
\begin{exa}
\label{exa:gstruct}
We consider three examples
\begin{enumerate}
\item[(i)]
Let $G = \lbrace e \rbrace$ be trivial, then there is only one possible $\chi \colon \lbrace e \rbrace \to O(n)$ taking $e \in G$ to $\mathrm{id}_{O(n)}$ and a $G$-structure is nothing but a trivializaton of the stabilized tangent bundle, i.e. an $n$-framing. Hence, we have an equivalence $\mathbf{Bord}_{n}^{\lbrace e \rbrace} \cong \mathbf{Bord}_{n}^{\mathrm{fr}}$.

\item[(ii)]
Let $G = SO(n)$ and $\chi \colon SO(n) \to O(n)$ the inclusion. Then a $G$-structure is a reduction of the structure group of the stabilized tangent bundle from $O(n)$ to $SO(n)$ which is an orientation of the bundle and thus basically an orientation of the manifold. Hence, we have an equivalence $\mathbf{Bord}_{n}^{SO(n)} \cong \mathbf{Bord}_{n}^{\mathrm{or}}$.

\item[(iii)]
Let $G = O(n)$ and $\chi \colon O(n) \to O(n)$ the identity. Then a $G$-structure is no change of the structure group at all and thus no structure on the manifold. Hence, we have an equivalence $\mathbf{Bord}_{n}^{O(n)} \cong \mathbf{Bord}_{n}$.
\end{enumerate}
\end{exa}
\begin{prf}
The details are left to the diligent reader.
\\
\phantom{proven}
\hfill
$\Box$
\end{prf}
Now let $G$ be a topological group acting on a topological space $Y$ and such that $G$ and $\mathrm{E}G$ are CW-complexes, which in particular can be arranged when $G$ is a compact Lie group. Then the homotopy type of the $G$-equivariant maps $\mathrm{mor}_{G}(\mathrm{E}G,Y)$ is independent of the choice of $\mathrm{E}G$ and is called the \textit{homotopy fixed point set (of the $G$-action on $Y$)}. We denote it $Y^{\mathrm{h}G}$. For an $\infty$-groupoid ${_{(\infty,0)}}\mathbf{C}$ which carries a $G$-action we likewise denote the $\infty$-groupoid $\mathrm{mor}_{G}(\mathrm{E}G,{_{(\infty,0)}}\mathbf{C})$ (defined as in the case of an $O(n)$-action) as ${_{(\infty,0)}}\mathbf{C}^{\mathrm{h}G}$. This allows us to formulate the following version of the cobordism hypothesis for manifolds with $G$-structures.
\\
\begin{cor}[Cobordism Hypothesis: $G$-Structure Version]
\label{cor:cobhypgstruct}
Let ${_{(\infty,n)}}\mathbf{C}$ be a symmetric monoidal $(\infty,n)$-category with duals. Further, let $G$ be a topological group such that $G$ and the total space $\mathrm{E}G$ of its universal bundle are CW-complexes and let $\chi \colon G \to O(n)$ be a continuous group homomorphism. Then using the $G$-action on the underlying $\infty$-groupoid $\mathrm{G}({_{(\infty,n)}}\mathbf{C})$ defined by $\chi$ through the $O(n)$-action there is an equivalence of $\infty$-groupoids
\begin{align*}
  \mathrm{func}^{\otimes,\mathrm{sym}}
  \left(
    \mathbf{Bord}_{n}^{G}
    ,
    {_{(\infty,n)}}\mathbf{C}
  \right)
  \to
  \left(
    \mathrm{G}({_{(\infty,n)}}\mathbf{C})
  \right)^{\mathrm{h}G}
\end{align*}
\end{cor}
\begin{prf}[Sketch]
As above, let $\zeta_{\chi}$ be the vector bundle associated with the universal bundle $\hat{\xi}_{G}$. From theorem \ref{thm:cobhypyzstruct} we know that
\begin{align*}
  \mathrm{func}^{\otimes,\mathrm{sym}}
  \left(
    \mathbf{Bord}_{n}^{G}
    ,
    {_{(\infty,n)}}\mathbf{C}
  \right)
\end{align*}
can be identified with
\begin{align*}
  \mathrm{mor}_{O(n)}
  \left(
    E_{\mathsf{OF}(\zeta_{\chi})}
    ,
    \mathrm{G}({_{(\infty,n)}}\mathbf{C})
  \right)
\end{align*}
and from the above discussion we know that $E_{\mathsf{OF}(\zeta_{\chi})}$ can be identified with $\mathrm{E}G \times_{G,\chi} O(n)$. Hence, in order to establish the assertion of this corollary we have to show the equivalence
\begin{align*}
  \mathrm{mor}_{O(n)}
  \left(
    \mathrm{E}G
    \times_{G,\chi}
    O(n)
    ,
    \mathrm{G}({_{(\infty,n)}}\mathbf{C})
  \right)
  &\cong
  \mathrm{mor}_{G}
  \left(
    \mathrm{E}G
    ,
    \mathrm{G}({_{(\infty,n)}}\mathbf{C})
  \right)
\end{align*}
where on the left we have the usual $O(n)$-action on $\mathrm{G}({_{(\infty,n)}}\mathbf{C})$ stemming from theorem \ref{thm:cobhypframed} and on the right we have the action induced by $\chi$. We only show that $\mathrm{mor}_{O(n)}(\mathrm{E}G \times_{G,\chi} O(n),Y)$ and $\mathrm{mor}_{G}(\mathrm{E}G,Y)$ are bijective for any space $Y$ with an $O(n)$-action and the corresponding $G$-action induced by $\chi$ and omit continuity properties of this bijection.
\\
Define $h_{O(n)} \in \mathrm{mor}_{O(n)}(\mathrm{E}G \times_{G,\chi} O(n),Y)$ for $h \in \mathrm{mor}_{G}(\mathrm{E}G,Y)$ by
\begin{align*}
  h_{O(n)}([p,o])
  &:=
  h(p)
  o
  \in
  Y
\end{align*}
for $[p,o] \in \mathrm{E}G \times_{G,\chi} O(n)$. This is well-defined as a map since for $[p_{1},o_{1}] = [p_{2},o_{2}] \in \mathrm{E}G \times_{G,\chi} O(n)$ there is some $g \in G$ such that
\begin{align*}
  (p_{2},o_{2})
  &=
  (p_{1},o_{1})
  g
  =
  (p_{1}g,\chi(g^{-1})o_{1})
\end{align*}
which implies
\begin{align*}
  h_{O(n)}([p_{2},o_{2}])
  &=
  h_{O(n)}([p_{1}g,\chi(g^{-1})o_{1}])
  \\
  &=
  h(p_{1}g)
  \chi(g^{-1})
  o_{1}
  \\
  &=
  h(p_{1})
  \chi(g)
  \chi(g^{-1})
  o_{1}
  \\
  &=
  h(p_{1})
  o_{1}
  \\
  &=
  h_{O(n)}([p_{1},o_{1}])
\end{align*}
Moreover, $h_{O(n)}$ is continuous as a composition of continuous maps and it is $O(n)$-equivariant,
\begin{align*}
  h_{O(n)}([p,o_{1}]o_{2})
  &=
  h_{O(n)}([p,o_{1}o_{2}])
  =
  h(p)
  o_{1}
  o_{2}
  =
  h_{O(n)}([p,o_{1}])
  o_{2}
\end{align*}
Conversely, for $h \in \mathrm{mor}_{O(n)}(\mathrm{E}G \times_{G,\chi} O(n),Y)$ define $h_{G} \in \mathrm{mor}_{G}(\mathrm{E}G,Y)$ by
\begin{align*}
  h_{G}(p)
  &:=
  h([p,\mathrm{id}_{O(n)}])
  \in
  Y
\end{align*}
for $p \in \mathrm{E}G$, where $\mathrm{id}_{O(n)}$ is the neutral element of $O(n)$. Then $h_{G}$ is again continuous and it is $G$-equivariant,
\begin{align*}
  h_{G}(pg)
  &=
  h([pg,\mathrm{id}_{O(n)}])
  \\
  &=
  h([p,\chi(g)])
  \\
  &=
  h([p,\mathrm{id}_{O(n)}]\chi(g))
  \\
  &=
  h([p,\mathrm{id}_{O(n)}])
  \chi(g)
  \\
  &=
  h_{G}(p)
  \chi(g)
\end{align*}
It is immediately checked that these constructions are inverse.
\\
To finish the proof a closer investigation of the topologies (inherited from the compact-open topologies of the full function spaces) of the two functions spaces and the continuity properties of the maps between them is needed to show that $\mathrm{mor}_{O(n)}(\mathrm{E}G \times_{G,\chi} O(n),Y)$ and $\mathrm{mor}_{G}(\mathrm{E}G,Y)$ indeed have the same homotopy type. Yet we will not do that here.
\\
\phantom{proven}
\hfill
$\Box$
\end{prf}
In the special case of the trivial group $G = \lbrace e \rbrace$ the total space of the universal bundle can be taken to be a singleton $\mathrm{E}G = \lbrace \bullet \rbrace$ so that $\mathrm{mor}_{G}(\mathrm{E}G,Y)$ can be identified with $Y$ in the obvious way and the compact-open topology agrees with the topology on $Y$ under this identification. Hence, we can conlude in this case that 
\begin{align*}
  \mathrm{mor}_{G}
  \left(
    \mathrm{E}G
    ,
    \mathrm{G}({_{(\infty,n)}}\mathbf{C})
  \right)
  &\cong
  \mathrm{G}({_{(\infty,n)}}\mathbf{C})
\end{align*}
so that corollary \ref{cor:cobhypgstruct} reduces to theorem \ref{thm:cobhypframed}.
