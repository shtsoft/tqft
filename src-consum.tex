\nocite{0a816f4c}
%%%
The last property we examine is about the connected sum, which is a way to produce a new connected $n$-manifold from two given connected $n$-manifolds. This means we can take two connected cobordisms and make a new one out of them. The property we consider then says that the linear map associated to the new cobordism by a TQFT can easily be calculated from the old ones, provided that a certain condition holds, namely that the state space on the $(n - 1)$-sphere $S^{n-1}$ is $1$-dimensional.
\\
We first sketch how to define this connected sum. For more details see e.g. \cite{wiki-map00000} in the article \href{http://www.map.mpim-bonn.mpg.de/Connected_sum}{Connected sum} and the references therein. Let $S_{1},S_{2}$ be objects in $\mathbf{Cob}_{n}$ and
\begin{align*}
  M_{1}
  \colon
  \emptyset
  &\to
  S_{1}
  ,\qquad
  M_{2}
  \colon
  S_{2}
  \to
  \emptyset
\end{align*}
be connected compact oriented cobordisms. To define the connected sum
\begin{align*}
  [M_{1}]
  \#
  [M_{2}]
\end{align*}
of the corresponding morphisms we cut out $n$-balls
\begin{align*}
  \mathbb{B}^{n}
  &:=
  \lbrace
    x
    \in
    \mathbb{R}^{n}
    \colon
    \Vert
    x
    \Vert
    \leq
    1
  \rbrace
\end{align*}
- or more precisely, we cut out embeddings of the $n$-ball, where the embedding into $M_{1}$ preserves orientation and the one into $M_{2}$ reverses orientation - from $M_{1}$ and $M_{2}$ to obtain cobordims
\begin{align*}
  \tilde{M_{1}}
  \colon
  S^{n-1}
  &\to
  S_{1}
  \qquad
  \text{and}
  \qquad
  \tilde{M_{2}}
  \colon
  S_{2}
  \to
  S^{n-1}
\end{align*}
Then we glue the resulting manifolds together along the resulting boundary spheres to obtain the new cobordism
\begin{align*}
  M_{1}
  \#
  M_{2}
  \colon 
  S_{2}
  &\to
  S_{1}
\end{align*}
Regarding the $n$-ball as cobordisms
\begin{align*}
  \mathbb{B}_{\dashv}^{n}
  \colon
  \emptyset
  &\to
  S^{n-1}
  \qquad
  \text{and}
  \qquad
  \mathbb{B}_{\vdash}^{n}
  \colon
  S^{n-1}
  \to
  \emptyset
\end{align*}
respectively (see figure \ref{fig:nballs} for an illustration), this means that we have
\begin{align*}
  [M_{1}]
  &=
  [\tilde{M_{1}}]
  \circ
  [\mathbb{B}_{\dashv}^{n}]
  ,\qquad
  [M_{2}]
  =
  [\mathbb{B}_{\vdash}^{n}]
  \circ
  [\tilde{M_{2}}]
\end{align*}
and the \textbf{connected sum} is defined by
\begin{align*}
  [M_{1}] \# [M_{2}]
  &:=
  [M_{1} \# M_{2}]
  :=
  [\tilde{M_{1}}]
  \circ
  [\tilde{M_{2}}]
\end{align*}
Using the disc theorem one can show that this construction is unique up to diffeomorphism, that is, does not depend on the chosen embeddings of the $n$-ball, so that the resulting morphism in $\mathbf{Cob}_{n}$ is well-defined. Moreover, the resulting morphism is connected which means that one and thus each cobordism representing the morphism is connected. The idea now is that when the state space of $S^{n-1}$ is $1$-dimensional then a TQFT yields, up to a constant, the same result for the composition of $M_{1}$ and $M_{2}$ and their connected sum. See figure \ref{fig:consumexa} for an illustration. This is because the state space for the empty $(n-1)$-manifold is $1$-dimensional, too, and linear maps between $1$-dimensional vector spaces cannot do more than multiplying by a constant when fixing bases.
\\
\begin{figure}[h!]
\centering
\begin{tikzpicture}[tqft/cobordism/.style={draw}]
  %left
  %top
  \pic[tqft,name=rtp,boundary separation=1.75cm,incoming boundary components=3,outgoing boundary components=1,offset=1,every incoming lower boundary component/.style={draw,ultra thin,dashed}];
  \node[at=(rtp-incoming boundary 2),below=14pt,font=\small]{$\tilde{M}_{2}$};
  \node[at=(rtp-outgoing boundary 1),left=15pt,font=\small]{$M_{2}$};
  \pic[tqft/cup,name=cup,at=(rtp-outgoing boundary 1),every incoming lower boundary component/.style={draw,ultra thin,dashed}];
  \node[at=(cup-incoming boundary 1),below=11pt,font=\small]{$\mathbb{B}_{\vdash}^{n}$};
  \node[at=(rtp-outgoing boundary 1),left=2cm,below=0.65cm,font=\small]{$M_{1} \circ M_{2}$};
  %bottom
  \pic[tqft/cap,name=cap,anchor={(1,0)},at=(rtp-outgoing boundary 1),every outgoing lower boundary component/.style={draw,ultra thin,dashed}];
  \node[at=(cap-outgoing boundary 1),above=11pt,font=\small]{$\mathbb{B}_{\dashv}^{n}$};
  \node[at=(cap-outgoing boundary 1),left=15pt,font=\small]{$M_{1}$};
  \pic[tqft/pair of pants,name=p,at=(cap-outgoing boundary 1),every incoming lower boundary component/.style={draw,ultra thin,dashed},every outgoing lower boundary component/.style={draw}];
  \node[at=(p-between outgoing 1 and 2),above=5pt,font=\small]{$\tilde{M}_{1}$};
  
  %right
  \node[at=(rtp-outgoing boundary 1),right=3.1cm,below=0.8cm]{$\to$};
  %top
  \pic[tqft,name=rtp2,anchor={(-2.5,-0.5)},at=(rtp-incoming boundary 1),boundary separation=1.75cm,incoming boundary components=3,outgoing boundary components=1,offset=1,every incoming lower boundary component/.style={draw,ultra thin,dashed}];
  \node[at=(rtp2-incoming boundary 2),below=14pt,font=\small]{$\tilde{M}_{2}$};
  \node[at=(rtp2-outgoing boundary 1),right=1.5cm,font=\small]{$M_{1} \# M_{2}$};
  %bottom
  \pic[tqft/pair of pants,name=p2,at=(rtp2-outgoing boundary 1),every incoming lower boundary component/.style={draw,ultra thin,dashed},every outgoing lower boundary component/.style={draw}];
  \node[at=(p2-between outgoing 1 and 2),above=5pt,font=\small]{$\tilde{M}_{1}$};

  %equation
  \node[at=(p-between outgoing 1 and 2),below=1.5cm,font=\small]{$Z([M_{1}] \circ [M_{2}])$};
  \node[at=(rtp-outgoing boundary 1),right=3.1cm,below=5.15cm]{$=$};
  \node[at=(p2-between outgoing 1 and 2),below=2.5cm,font=\small]{$\mathrm{const} \cdot Z([M_{1}] \# [M_{2}])$};
\end{tikzpicture}
\caption{Illustration of the calculation of a TQFT for the connected sum}
\label{fig:consumexa}
\end{figure}
\\
More precisely, we have the following
\\
\begin{thm}
\label{thm:consum}
Let
\begin{align*}
  Z
  \colon
  \mathbf{Cob}_{n}
  &\to
  \mathbf{Vec}_{K}
\end{align*}
be a TQFT with
\begin{align*}
  \dim(Z(S^{n-1}))
  &=
  1
\end{align*}
Then
\begin{align*}
\Phi^{-1}(Z([S^{n}])(\Phi(1))) \neq 0
\end{align*}
and for $S_{1},S_{2} \in \mathrm{ob}_{\mathbf{Cob}_{n}}$ and connected
\begin{align*}
  [M_{1}]
  \in
  \mathrm{mor}_{\mathbf{Cob}_{n}}(\emptyset,S_{1})
  ,\qquad
  [M_{2}]
  \in
  \mathrm{mor}_{\mathbf{Cob}_{n}}(S_{2},\emptyset)
\end{align*}
we have
\begin{align*}
  Z([M_{1}] \# [M_{2}])
  &=
  \frac{1}{\Phi^{-1}(Z([S^{n}])(\Phi(1)))}
  Z([M_{1}])
  \circ
  Z([M_{2}])
\end{align*}
\end{thm}
\begin{prf}
We keep using the notation from the definition of the connected sum above. First note that
\begin{align*}
  [\mathbb{B}_{\vdash}^{n}]
  \circ
  [\mathbb{B}_{\dashv}^{n}]
  \colon
  \emptyset
  &\to
  \emptyset
\end{align*}
can be represented by the $n$-sphere $S^{n} \colon \emptyset \to \emptyset$, as illustrated in figure \ref{fig:nballs}, which means
\begin{align*}
  Z([\mathbb{B}_{\vdash}^{n}])
  \circ
  Z([\mathbb{B}_{\dashv}^{n}])
  &=
  Z([\mathbb{B}_{\vdash}^{n}] \circ [\mathbb{B}_{\dashv}^{n}])
  \\
  &=
  Z([S^{n}])
\end{align*}
\begin{figure}[h!]
\centering
\begin{tikzpicture}[every node/.style={scale=1.5},tqft/cobordism/.style={draw}]
  %left
  %top
  \pic[tqft/cap,name=bu,cobordism height=8em,circle x radius=1.8em,circle y radius=0.7em,every outgoing boundary component/.style={draw,ultra thin,dashed}];
  \node[at=(bu-outgoing boundary 1),font=\small]{$S^{n-1}$};
  \node[at=(bu-outgoing boundary 1),above=1.9em,font=\small]{$\mathbb{B}_{\dashv}^{n}$};
  %bottom
  \pic[tqft/cup,name=bd,at=(bu-outgoing boundary 1),cobordism height=8em,circle x radius=1.8em,circle y radius=0.7em,every incoming boundary component/.style={draw,ultra thin,dashed}];
  \node[at=(bd-incoming boundary 1),below=2.1em,font=\small]{$\mathbb{B}_{\vdash}^{n}$};

  %right
  \node[at=(bu-outgoing boundary 1),right=1.3cm,font=\small]{$\cong \quad S^{n}$};
\end{tikzpicture}
\caption{Gluing two $n$-balls together yields an $n$-sphere}
\label{fig:nballs}
\end{figure}
\\
Since $Z(S^{n-1})$ is $1$-dimensional we can choose a basis element, say $e$, and then we have
\begin{align*}
  Z([\mathbb{B}_{\dashv}^{n}])(\Phi(1))
  &=
  b_{\dashv}
  e
  ,\qquad
  \Phi^{-1}
  \left(
    Z([\mathbb{B}_{\vdash}^{n}])(e)
  \right)
  =
  b_{\vdash}
  ,\qquad
  Z([\tilde{M_{2}}])(v)
  =
  c(v)
  e
\end{align*}
for certain $b_{\dashv},b_{\vdash},c(v) \in K$ and all $v \in Z(S_{2})$. With this and the linearity of all involved maps we can easily calculate
\begin{align*}
  &
  \left(
    Z([\tilde{M_{1}}])
    \circ
    Z([\mathbb{B}_{\dashv}^{n}])
    \circ
    Z([\mathbb{B}_{\vdash}^{n}])
    \circ
    Z([\tilde{M_{2}}])
  \right)
  (v)
  \\
  &=
  c(v)
  \left(
    Z([\tilde{M_{1}}])
    \circ
    Z([\mathbb{B}_{\dashv}^{n}])
    \circ
    Z([\mathbb{B}_{\vdash}^{n}])
  \right)
  (e)
  \\
  &=
  b_{\vdash}
  c(v)
  \left(
    Z([\tilde{M_{1}}])
    \circ
    Z([\mathbb{B}_{\dashv}^{n}])
  \right)
  (\Phi(1))
  \\
  &=
  b_{\dashv}
  b_{\vdash}
  c(v)
  Z([\tilde{M_{1}}])(e)
  \\
  &=
  b_{\dashv}
  \Phi^{-1}
  \left(
    Z([\mathbb{B}_{\vdash}^{n}])(e)
  \right)
  Z([\tilde{M_{1}}])(c(v)e)
  \\
  &=
  \Phi^{-1}
  \left(
    Z([\mathbb{B}_{\vdash}^{n}])(b_{\dashv}e)
  \right)
  \left(
    Z([\tilde{M_{1}}])
    \circ
    Z([\tilde{M_{2}}])
  \right)
  (v)
  \\
  &=
  \Phi^{-1}
  \left(
    \left(
      Z([\mathbb{B}_{\vdash}^{n}])
      \circ
      Z([\mathbb{B}_{\dashv}^{n}])
    \right)
    (\Phi(1))
  \right)
  \left(
    Z([\tilde{M_{1}}])
    \circ
    Z([\tilde{M_{2}}])
  \right)
  (v)
\end{align*}
This shows that
\begin{align*}
  Z([M_{1}])
  \circ
  Z([M_{2}])
  &=
  Z([\tilde{M_{1}}])
  \circ
  Z([\mathbb{B}_{\dashv}^{n}])
  \circ
  Z([\mathbb{B}_{\vdash}^{n}])
  \circ
  Z([\tilde{M_{2}}])
  \\
  &=
  \Phi^{-1}
  \left(
    \left(
      Z([\mathbb{B}_{\vdash}^{n}])
      \circ
      Z([\mathbb{B}_{\dashv}^{n}])
    \right)
    (\Phi(1))
  \right)
  \left(
    Z([\tilde{M_{1}}])
    \circ
    Z([\tilde{M_{2}}])
  \right)
  \\
  &=
  \Phi^{-1}
  \left(
    Z([S^{n}])(\Phi(1))
  \right)
  Z([M_{1}] \# [M_{2}])
\end{align*}
If we can show that
\begin{align*}
  \Phi^{-1}(Z([S^{n}])(\Phi(1)))
  &\neq
  0
\end{align*}
then we are done. To this end suppose
\begin{align*}
  \Phi^{-1}(Z([S^{n}])(\Phi(1)))
  &=
  0
\end{align*}
then the above equation implies
\begin{align*}
  Z([M_{1}]) \circ Z([M_{2}])
  &=
  0
\end{align*}
for all $S_{1},S_{2} \in \mathrm{ob}_{\mathbf{Cob}_{n}}$ and connected
\begin{align*}
  [M_{1}]
  \in
  \mathrm{mor}_{\mathbf{Cob}_{n}}(\emptyset,S_{1})
  ,\qquad
  [M_{2}]
  \in
  \mathrm{mor}_{\mathbf{Cob}_{n}}(S_{2},\emptyset)
\end{align*}
To bring this to a contradiction let
\begin{align*}
  [M_{1}]
  &:=
  \mathrm{coev}_{\overline{S^{n-1}}}
  \qquad
  \text{and}
  \qquad
  [M_{2}]
  :=
  \mathrm{ev}_{S^{n-1}}
\end{align*}
which are both clearly connected. Consider the morphisms $X$ and $Y$ defined by making the following diagrams commute
\begin{equation*}
\hspace{-4em}
\begin{tikzcd}[row sep=5em,column sep=6em]
  S^{n-1}
  \sqcup
  (\overline{S^{n-1}} \sqcup S^{n-1})
  \ar{r}{X}
  \ar{d}[swap]{\mathsf{A}^{-1}(S^{n-1},\overline{S^{n-1}},S^{n-1})}
  &
  S^{n-1}
  \\
  (S^{n-1} \sqcup \overline{S^{n-1}})
  \sqcup
  S^{n-1}
  \ar{r}{\mathrm{ev}_{\overline{S^{n-1}}} \sqcup \mathrm{id}_{S^{n-1}}}
  &
  \emptyset
  \sqcup
  S^{n-1}
  \ar{u}[swap]{\mathsf{L}(S^{n-1})}
\end{tikzcd}
\end{equation*}
\begin{equation*}
\hspace{4em}
\begin{tikzcd}[row sep=5em,column sep=6em]
  S^{n-1}
  \ar{r}{Y}
  \ar{d}[swap]{\mathsf{L}^{-1}(S^{n-1})}
  &
  S^{n-1}
  \sqcup
  (\overline{S^{n-1}} \sqcup S^{n-1})
  \\
  \emptyset
  \sqcup
  S^{n-1}
  \ar{r}{\mathrm{coev}_{S^{n-1}} \sqcup \mathrm{id}_{S^{n-1}}}
  &
  (S^{n-1} \sqcup \overline{S^{n-1}})
  \sqcup
  S^{n-1}
  \ar{u}[swap]{\mathsf{A}(S^{n-1},\overline{S^{n-1}},S^{n-1})}
\end{tikzcd}
\end{equation*}
where we used $\overline{\overline{S^{n-1}}} = S^{n-1}$ in the first diagram. We will show that
\begin{align}
\label{diffcyl}
  X
  \circ
  \left(
    \mathrm{id}_{S^{n-1}}
    \sqcup
    \left(
      \mathrm{coev}_{\overline{S^{n-1}}}
      \circ
      \mathrm{ev}_{S^{n-1}}
    \right)
  \right)
  \circ
  Y
  &=
  \mathrm{id}_{S^{n-1}}
\end{align}
See figure \ref{fig:s1contr} for an illustration with the associativity and unit cylinders suppressed.
\\
\begin{figure}[h!]
\centering
\begin{tikzpicture}[tqft/cobordism/.style={draw}]
  %left
  %coevaluation
  \pic[tqft,name=co,cobordism height=3cm,boundary separation=1.75cm,incoming boundary components=0,outgoing boundary components=2,every outgoing boundary component/.style={draw,ultra thin,dashed}];
  \node[at=(co-between outgoing 1 and 2),above=3pt,font=\small]{$\mathrm{coev}_{S^{n-1}}$};
  %evaluation
  \pic[tqft,name=ev,cobordism height=3cm,boundary separation=1.75cm,incoming boundary components=2,outgoing boundary components=0,anchor=incoming boundary 1,at=(co-outgoing boundary 2),every incoming lower boundary component/.style={draw,ultra thin,dashed}];
  \node[at=(ev-between incoming 1 and 2),below=4pt,font=\small]{$\mathrm{ev}_{S^{n-1}}$};
  %cylinders
  \pic[tqft/cylinder,name=c1,cobordism height=1.6cm,anchor=outgoing boundary 1,at=(ev-incoming boundary 2),every incoming lower boundary component/.style={draw,ultra thin,dashed},every outgoing boundary component/.style={draw,ultra thin,dashed}];
  \node[at=(c1-incoming boundary 1),above=3pt,font=\small]{$S^{n-1}$};
  \node[at=(c1-between first incoming and first outgoing),right=-2pt,font=\small]{$\mathrm{id}_{S^{n-1}}$};
  \pic[tqft/cylinder,name=c2,cobordism height=1.6cm,anchor=incoming boundary 1,at=(co-outgoing boundary 1),every incoming lower boundary component/.style={draw,ultra thin,dashed},every outgoing boundary component/.style={draw,ultra thin,dashed}];
  \node[at=(c2-between first incoming and first outgoing),right=-2pt,font=\small]{$\mathrm{id}_{S^{n-1}}$};
  \pic[tqft/cylinder,name=c3,cobordism height=1.6cm,anchor=incoming boundary 1,at=(c2-outgoing boundary 1),every incoming lower boundary component/.style={draw,ultra thin,dashed},every outgoing boundary component/.style={draw,ultra thin,dashed}];
  \node[at=(c3-between first incoming and first outgoing),right=-2pt,font=\small]{$\mathrm{id}_{S^{n-1}}$};
  %opposite evaluation
  \pic[tqft,name=evop,cobordism height=3cm,boundary separation=1.75cm,incoming boundary components=2,outgoing boundary components=0,anchor=incoming boundary 1,at=(c3-outgoing boundary 1),every incoming lower boundary component/.style={draw,ultra thin,dashed}];
  \node[at=(evop-between incoming 1 and 2),below=4pt,font=\small]{$\mathrm{ev}_{\overline{S^{n-1}}}$};
  %opposite coevaluation
  \pic[tqft,name=coop,cobordism height=3cm,boundary separation=1.75cm,incoming boundary components=0,outgoing boundary components=2,anchor=outgoing boundary 1,at=(evop-incoming boundary 2),every outgoing boundary component/.style={draw,ultra thin,dashed}];
  \node[at=(coop-between outgoing 1 and 2),above=1pt,font=\small]{$\mathrm{coev}_{\overline{S^{n-1}}}$};
  %cylinder
  \pic[tqft/cylinder,name=c4,cobordism height=1.6cm,anchor=incoming boundary 1,at=(coop-outgoing boundary 2),every incoming lower boundary component/.style={draw,ultra thin,dashed},every outgoing boundary component/.style={draw}];
  \node[at=(c4-outgoing boundary 1),below=4pt,font=\small]{$S^{n-1}$};
  \node[at=(c4-between first incoming and first outgoing),right=-2pt,font=\small]{$\mathrm{id}_{S^{n-1}}$};

  %right
  \node[at=(c2-outgoing boundary 1),right=5.5cm]{$\cong$};
  %top
  \pic[tqft/cylinder,name=c5,cobordism height=2.8cm,anchor={(-1.3,-0.175)},at=(c1-incoming boundary 1),every incoming lower boundary component/.style={draw,ultra thin,dashed},every outgoing boundary component/.style={draw,ultra thin,dashed}];
  \node[at=(c5-incoming boundary 1),above=3pt,font=\small]{$S^{n-1}$};
  \node[at=(c5-between first incoming and first outgoing),right=-2pt,font=\small]{$\mathrm{id}_{S^{n-1}}$};
  %bottom
  \pic[tqft/cylinder,name=c6,cobordism height=2.8cm,anchor=incoming boundary 1,at=(c5-outgoing boundary 1),every incoming lower boundary component/.style={draw,ultra thin,dashed},every outgoing boundary component/.style={draw}];
  \node[at=(c6-outgoing boundary 1),below=4pt,font=\small]{$S^{n-1}$};
  \node[at=(c6-between first incoming and first outgoing),right=-2pt,font=\small]{$\mathrm{id}_{S^{n-1}}$};
\end{tikzpicture}
\caption{Illustration of equation \eqref{diffcyl}}
\label{fig:s1contr}
\end{figure}
\newpage
By definition we have the outer perimeter of the following commuting diagram. The lower part follows from the functoriality of $\sqcup$, the lower left part follows from diagram (LD1) for $\mathrm{coev}_{S^{n-1}}$ and $\mathrm{ev}_{S^{n-1}}$ and the lower right part follows from diagram (LD2) for $\mathrm{coev}_{\overline{S^{n-1}}}$ and $\mathrm{ev}_{\overline{S^{n-1}}}$. The central part obviously commutes and hence so does the upper part proving equation \eqref{diffcyl}.
\begin{equation*}
\begin{tikzcd}[row sep=5em,column sep=6em,font=\footnotesize,every label/.append style={font=\tiny}]
  S^{n-1}
  \sqcup
  (\overline{S^{n-1}} \sqcup S^{n-1})
  \ar{rr}{\mathrm{id}_{S^{n-1}} \sqcup (\mathrm{coev}_{\overline{S^{n-1}}} \circ \mathrm{ev}_{S^{n-1}})}
  &
  &
  S^{n-1}
  \sqcup
  (\overline{S^{n-1}} \sqcup S^{n-1})
  \ar{d}{X}
  \\
  S^{n-1}
  \ar{u}{Y}
  \ar{rr}{\mathrm{id}_{S^{n-1}}}
  \ar{rddd}{\mathsf{R}^{-1}(S^{n-1})}
  \ar{d}[swap]{\mathsf{L}^{-1}(S^{n-1})}
  &
  &
  S^{n-1}
  \\
  \emptyset
  \sqcup
  S^{n-1}
  \ar{d}[swap]{\mathrm{coev}_{S^{n-1}} \sqcup \mathrm{id}_{S^{n-1}}}
  &
  &
  \emptyset
  \sqcup
  S^{n-1}
  \ar{u}[swap]{\mathsf{L}(S^{n-1})}
  \\
  (S^{n-1} \sqcup \overline{S^{n-1}})
  \sqcup
  S^{n-1}
  \ar{d}[swap]{\mathsf{A}(S^{n-1},\overline{S^{n-1}},S^{n-1})}
  &
  &
  (S^{n-1} \sqcup \overline{S^{n-1}})
  \sqcup
  S^{n-1}
  \ar{u}[swap]{\mathrm{ev}_{\overline{S^{n-1}}} \sqcup \mathrm{id}_{S^{n-1}}}
  \\
  S^{n-1}
  \sqcup
  (\overline{S^{n-1}} \sqcup S^{n-1})
  \ar{r}{\mathrm{id}_{S^{n-1}} \sqcup \mathrm{ev}_{S^{n-1}}}
  \ar[bend right]{rr}[yshift=4pt]{\mathrm{id}_{S^{n-1}} \sqcup (\mathrm{coev}_{\overline{S^{n-1}}} \circ \mathrm{ev}_{S^{n-1}})}
  &
  S^{n-1}
  \sqcup
  \emptyset
  \ar{uuur}{\mathsf{R}(S^{n-1})}
  \ar{r}{\mathrm{id}_{S^{n-1}} \sqcup \mathrm{coev}_{\overline{S^{n-1}}}}
  &
  S^{n-1}
  \sqcup
  (\overline{S^{n-1}} \sqcup S^{n-1})
  \ar{u}[swap]{\mathsf{A}^{-1}(S^{n-1},\overline{S^{n-1}},S^{n-1})}
\end{tikzcd}
\end{equation*}
Now we apply $Z$ to equation \eqref{diffcyl} which yields zero for the left-hand side since
\begin{align*}
  &
  Z
  \left(
    \mathrm{id}_{S^{n-1}}
    \sqcup
    \left(
      \mathrm{coev}_{\overline{S^{n-1}}}
      \circ
      \mathrm{ev}_{S^{n-1}}
    \right)
  \right)
  \\
  &=
  \mathsf{H}
  \left(
    S^{n-1}
    ,
    \overline{S^{n-1}} \sqcup S^{n-1}
  \right)
  \circ
  \left(
    Z(\mathrm{id}_{S^{n-1}})
    \otimes
    \left(
      Z(\mathrm{coev}_{\overline{S^{n-1}}})
      \circ
      Z(\mathrm{ev}_{S^{n-1}})
    \right)
  \right)
  \circ
  \mathsf{H}^{-1}
  \left(
    S^{n-1}
    ,
    \overline{S^{n-1}} \sqcup S^{n-1}
  \right)
  \\
  &=
  \mathsf{H}
  \left(
    S^{n-1}
    ,
    \overline{S^{n-1}} \sqcup S^{n-1}
  \right)
  \circ
  \left(
    \mathrm{id}_{Z(S^{n-1})}
    \otimes
    0
  \right)
  \circ
  \mathsf{H}^{-1}
  \left(
    S^{n-1}
    ,
    \overline{S^{n-1}} \sqcup S^{n-1}
  \right)
  \\
  &=
  0
\end{align*}
but for the right-hand side we obtain $\mathrm{id}_{Z(S^{n-1})}$ which is not zero since $\dim(Z(S^{n-1})) = 1$. This contradiction shows that
\begin{align*}
  \Phi^{-1}(Z(S^{n})(\Phi(1)))
  &\neq
  0
\end{align*}
and finishes our proof.
\\
\phantom{proven}
\hfill
$\square$
\end{prf}
Thus if the state space of the sphere $S^{n-1}$ for a TQFT is $1$-dimensional we have another way to calculate the linear map associated to a cobordism provided it can be written as the connected sum of two connected cobordisms where one has empty in-boundary and one has empty out-boundary.
