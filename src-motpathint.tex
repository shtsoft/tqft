In this chapter we will derive some properties of the path integral formulation of quantum field theories and examine how these properties lead to the functorial definition of TQFTs. Path integrals are an important tool in quantum field theory. Even though they cannot be rigorously defined in most cases, they yield physically sensible results and often are the standard way to do calculations in quantum field theory today. Despite the lack of mathematical precision, path integrals give a good intuition about QFTs, which is why they are used here to motivate the functorial description of topological quantum field theories.
\\
In section \ref{sec:piform} we briefly introduce how path integrals are used in QFT. The treatment here is rather basic and superficial so it is not necessary to know some QFT to understand it but it is certainly helpful. Section \ref{sec:pitqfts} then is about extracting some properties of path integrals which serve as a motivation for the categorical definition of ordinary TQFTs as given in the previous chapter \ref{chap:defordtqft}.
