%\nocite{cc6d78b5}
%\nocite{dfcdc48c}
%%%
So far we have seen the categorical definition of ordinary TQFTs and some of their basic properties. Now one may wonder how many such TQFTs exist and whether they can be classified in some, more or less, efficient way. For 1-dimensional TQFTs it is easy to give such a description very explicitly and for 2-dimensional TQFTs it is not too difficult either to find a classification in terms of algebraic objects. With increasing dimension however things become more and more complicated and one is lead to extended TQFTs which involve a more sophisticated version of the cobordism category.
\\
As already mentioned in the motivation in chapter \ref{chap:motqft} these extended TQFTs are what one really aims for from a physical point of view because they allow to cut spacetime not only in the time direction but also in the spatial directions. Thus they do not break general covariance as ordinary TQFTs do.
\\
The cobordism hypothesis - formulated in its original incarnation by Baez and Dolan in \cite{cc6d78b5} - now basically states that the more sophisticated cobordism category used for extended TQFTs has a rather simple algebraic description and thereby gives sort of a classification for extended TQFTs.
\\
These notions are naturally cast in the language of higher categories which first needs to be made precise. As this is rather involved we will treat higher categories on an informal level here which nevertheless should capture the idea.
\\
Lurie gives a formulation for the cobordism hypothesis and a rather detailed sketch of the proof in \cite{dfcdc48c}. Even though a fully detailed account is missing the proof is widely accepted by the experienced community.
\\\\
This part of the present work is based on this expository paper \cite{dfcdc48c} by Lurie and the objective here is to sufficiently prepare the reader to understand the formulation of the cobordism hypothesis as given there and how one arrives at this formulation. We do not, however, discuss the proof.
\\\\
There are again four chapters in this part. The first one is chapter \ref{chap:prelim2} where we start with some preliminaries we need in the following chapters. In chapter \ref{chap:lowdimtqft} we discuss low-dimensional ordinary TQFTs in order to see how one is led to extended TQFTs. The subsequent chapter \ref{chap:extcob} is about extending the ordinary category of cobordims to obtain the higher category of cobordisms mentioned above. Finally, in chapter \ref{chap:formcobhyp} we formulate different versions of the cobordism hypothesis.
